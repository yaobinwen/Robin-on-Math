\documentclass[12pt, letterpaper, oneside]{book}
\usepackage{amsmath}
\usepackage{amssymb}
\usepackage{empheq}
\usepackage[letterpaper, textwidth=7.5in, textheight=8in]{geometry}
\usepackage{csquotes}
\usepackage{parskip}

\title{Exercises in \textit{Linear Algebra Done Right}}
\author{Yaobin Wen}
\date{February 2023}

\begin{document}

\maketitle
\tableofcontents

% =============================================================================
%
% Chapter 1: Vector Spaces
%
% =============================================================================

\chapter{Vector Spaces}

% =============================================================================
\section{EXERCISES 1.A}
% =============================================================================

% ******************************
\subsection{Question 1}
% ******************************

Description: Suppose $a$ and $b$ are real numbers, not both 0. Find real numbers $c$ and $d$
such that

\[
  \frac{1}{(a+bi)} = c + di
\]

Answer:

$\frac{1}{(a+bi)} = c + di \Rightarrow (a+bi)(c+di) = 1 \Rightarrow (ac-bd) + (ad + bc)i = 1$

$\therefore$ We have:

\begin{empheq}[left=\empheqlbrace]{align}
  &ac - bd = 1 \\
  &ad + bc = 0
\end{empheq}

Because $a$ and $b$ are not both 0, there can be three cases.

Case 1: $a = 0$, $b \neq 0$

\[(1.1) \ and \ (1.2)\]

$\Rightarrow$

\begin{empheq}[left=\empheqlbrace]{align*}
  -bd &= 1 \\
  bc &= 0
\end{empheq}

$\Rightarrow$

\begin{empheq}[left=\empheqlbrace]{align*}
  &c = 0 \\
  &d = -\frac{1}{b}
\end{empheq}

Case 2: $a \neq 0$, $b = 0$

\[(1.1) \ and \ (1.2)\]

$\Rightarrow$

\begin{empheq}[left=\empheqlbrace]{align*}
  &ac = 1 \\
  &ad = 0
\end{empheq}

$\Rightarrow$

\begin{empheq}[left=\empheqlbrace]{align*}
  &c = \frac{1}{a} \\
  &d = 0
\end{empheq}

Case 3: $a \neq 0$, $b \neq 0$

\[(1.1) \ and \ (1.2)\]

$\Rightarrow$

\begin{empheq}[left=\empheqlbrace]{align*}
  &c = \frac{a}{a^2+b^2} \\
  &d = -\frac{b}{a^2+b^2}
\end{empheq}

(Done)

% ******************************
\subsection{Question 2}
% ******************************

Description: Show that \[ \frac{-1 + \sqrt{3}i}{2} \] is a cube root of $1$
(meaning that its cube equals $1$).

Solution:

\begin{equation*}
  \begin{split}
    (\frac{-1 + \sqrt{3}i}{2})^3
    & = (\frac{-1 + \sqrt{3}i}{2})^2 \cdot \frac{-1 + \sqrt{3}i}{2} \\
    & = \frac{1}{4}\bigl[ (-1)^2 - \sqrt{3}i - \sqrt{3}i + 3i^2 \bigr] \cdot
      \frac{-1 + \sqrt{3}i}{2} \\
    & = \frac{1}{4}\bigl[ 1 - 2\sqrt{3}i - 3 \bigr] \cdot
      \frac{-1 + \sqrt{3}i}{2} \\
    & = -\frac{1 + \sqrt{3}i}{2} \cdot \frac{-1 + \sqrt{3}i}{2} \\
    & = -\frac{1}{4}\bigl(-1 + \sqrt{3}i - \sqrt{3}i + 3i^2 \bigr) \\
    & = -\frac{1}{4} \cdot (-1 - 3) \\
    & = -\frac{1}{4} \cdot (-4) \\
    & = 1
  \end{split}
\end{equation*}

$\therefore$: $\frac{-1 + \sqrt{3}i}{2}$ is a cube root of $1$.

% ******************************
\subsection{Question 3}
% ******************************

TODO

% ******************************
\subsection{Question 4}
% ******************************

TODO

% ******************************
\subsection{Question 5}
% ******************************

TODO

% ******************************
\subsection{Question 6}
% ******************************

TODO

% ******************************
\subsection{Question 7}
% ******************************

TODO

% ******************************
\subsection{Question 8}
% ******************************

TODO

% ******************************
\subsection{Question 9}
% ******************************

TODO

% ******************************
\subsection{Question 10}
% ******************************

TODO

% ******************************
\subsection{Question 11}
% ******************************

Explain why there does not exist $\lambda \in \mathbf{C}$ such that
\[\lambda(2-3i, 5+4i, -6+7i) = (12-5i, 7+22i, -32-9i) \]

Solution:

According to the definition of $\mathbf{F}^n$, $(2-3i, 5+4i, -6+7i)$ and
$(12-5i, 7+22i, -32-9i)$ are two lists when $n=3$.

Because $\lambda \in \mathbf{C}$ and $\mathbf{F}$ can denote $\mathbf{C}$,
we have $\lambda \in \mathbf{F}$.

Therefore, the expression $\lambda(2-3i, 5+4i, -6+7i)$ is a \textit{scalar
multiplication} on $\mathbf{F}^3$, so we have:

\[
  \lambda(2-3i, 5+4i, -6+7i) = (\lambda(2-3i), \lambda(5+4i), \lambda(-6+7i))
\]

\begin{equation*}
  \begin{split}
    \lambda(2-3i, 5+4i, -6+7i)
    & = (\lambda(2-3i), \lambda(5+4i), \lambda(-6+7i)) \\
    & = (2\lambda - 3\lambda i, 5\lambda + 4\lambda i, -6\lambda + 7\lambda i)
  \end{split}
\end{equation*}

To make it equal to $(12-5i, 7+22i, -32-9i)$, we must have:

\begin{empheq}[left=\empheqlbrace]{align*}
  2\lambda - 3\lambda i & = 12-5i \\
  5\lambda + 4\lambda i & = 7+22i \\
  -6\lambda + 7\lambda i & = -32-9i
\end{empheq}

This can be further reduced as:

\begin{empheq}[left=\empheqlbrace]{align*}
  2\lambda & = 12 \\
  -3\lambda &= -5 \\
  5\lambda & = 7 \\
  4\lambda & = 22 \\
  -6\lambda & = -32 \\
  7\lambda & = -9
\end{empheq}

Unfortunately, we can't find a single $\lambda$ that satisfies all of the
equations above. Therefore, such a $\lambda$ doesn't exist.

% ******************************
\subsection{Question 12}
% ******************************

Show that $(x + y) + z = x + (y + z)$ for all $x, y, z \in \mathbf{F}^n$.

Solution:

Because $x, y, z \in \mathbf{F}^n$, according to the definition of
$\mathbf{F}^n$, we know that:

\[ x = (x_1, \ldots, x_n), x_j \in \mathbf{F}, j = 1, \ldots, n \]
\[ y = (y_1, \ldots, y_n), y_j \in \mathbf{F}, j = 1, \ldots, n \]
\[ z = (z_1, \ldots, z_n), z_j \in \mathbf{F}, j = 1, \ldots, n \]

Because $x_j, y_j, z_j \in \mathbf{F}$, their addition satisfies the property
of associativity, i.e.,

\[ (a + b) + c = a + (b + c), a, b, c \in \mathbf{F} \]

Therefore, according to the definition of addition on $\mathbf{F}^n$, we have:

\begin{equation*}
  \begin{split}
    (x + y) + z
    & = \bigl((x_1, \ldots, x_n) + (y_1, \ldots, y_n)\bigr) +
      (z_1, \ldots, z_n) \\
    & = (x_1 + y_1, \ldots, x_n + y_n) + (z_1, \ldots, z_n) \\
    & = \bigl((x_1 + y_1), \ldots, (x_n + y_n)\bigr) + (z_1, \ldots, z_n) \\
    & = \bigl((x_1 + y_1) + z_1, \ldots, (x_n + y_n) + z_n\bigr) \\
    & = \bigl(x_1 + (y_1 + z_1), \ldots, x_n + (y_n + z_n)\bigr) \\
    & = (x_1, \ldots, x_n) + \bigl((y_1 + z_1), \ldots, (y_n + z_n)\bigr) \\
    & = (x_1, \ldots, x_n) + \bigl(y_1 + z_1, \ldots, y_n + z_n\bigr) \\
    & = (x_1, \ldots, x_n) + \bigl((y_1, \ldots, y_n) +
      (z_1, \ldots, z_n)\bigr) \\
    & = x + (y + z)
  \end{split}
\end{equation*}

% ******************************
\subsection{Question 13}
% ******************************

Show that $(ab)x = a(bx)$ for all $x \in \mathbf{F}^n$ and all $a, b \in
\mathbf{F}$

Solution:

Because $x \in \mathbf{F}^n$, according to the definition of $\mathbf{F}^n$, we
know that:

\[ x = (x_1, \ldots, x_n), x_j \in \mathbf{F}, j = 1, \ldots, n \]

Because $a, b, x_j \in \mathbf{F}$, their multiplication satisfies the property
of associativity, i.e.,

\[ (ab)x_j = a(b x_j), a, b, x_j \in \mathbf{F} \]

Because $a, b \in \mathbf{F}$, we have $ab \in \mathbf{F}$. Therefore, $(ab)x$
is the case of scalar multiplication on $\mathbf{F}^n$. Therefore, according to
the definition of scalar multiplication on $\mathbf{F}^n$, we have:

\begin{equation*}
  \begin{split}
    (ab)x
    & = (ab)(x_1, \ldots, x_n) \\
    & = \bigl((ab)x_1, \ldots, (ab)x_n\bigr) \\
    & = \bigl(a(b x_1), \ldots, a(b x_n)\bigr) \\
    & = a(b x_1, \ldots, b x_n) \\
    & = a\bigl(b(x_1, \ldots, x_n)\bigr) \\
    & = a(bx)
  \end{split}
\end{equation*}

% ******************************
\subsection{Question 14}
% ******************************

Show that $1x = x$ for all $x \in \mathbf{F}^n$.

Solution:

Because we have never defined the multiplication of two elements in
$\mathbf{F}^n$ throughout the entire chapter, and because so far we have only
defined scalar multiplication on $\mathbf{F}^n$, we have $1$ must be a scalar
in $\mathbf{F}$.

Because $x \in \mathbf{F}^n$, according to the definition of $\mathbf{F}^n$, we
know that:

\[ x = (x_1, \ldots, x_n), x_j \in \mathbf{F}, j = 1, \ldots, n \]

Because $1$ is the multiplicative identity in $\mathbf{F}$, we know that

\[ 1a = a, \forall a \in \mathbf{F} \]

Therefore, according to the definition of scalar multiplication on
$\mathbf{F}^n$, we have:

\begin{equation*}
  \begin{split}
    1x
    & = 1(x_1, \ldots, x_n) \\
    & = (1 x_1, \ldots, 1 x_n) \\
    & = (x_1, \ldots, x_n) \\
    & = x
  \end{split}
\end{equation*}

% ******************************
\subsection{Question 15}
% ******************************

Show that $\lambda(x + y) = \lambda x + \lambda y$ for all $\lambda \in
\mathbf{F}$ and all $x, y \in \mathbf{F}^n$.

Solution:

Because $x, y \in \mathbf{F}^n$, according to the definition of $\mathbf{F}^n$,
we know that:

\[ x = (x_1, \ldots, x_n), x_j \in \mathbf{F}, j = 1, \ldots, n \]
\[ y = (y_1, \ldots, y_n), y_j \in \mathbf{F}, j = 1, \ldots, n \]

Because $\lambda, x_j, y_j \in \mathbf{F}$, the distributive property holds:

\[ \lambda(x_j + y_j) = \lambda x_j + \lambda y_j \]

Therefore, according to the addition and scalar multiplication on
$\mathbf{F}^n$, we have:

\begin{equation*}
  \begin{split}
    \lambda(x + y)
    & = \lambda \bigl((x_1, \ldots, x_n) + (y_1, \ldots, y_n)\bigr) \\
    & = \lambda (x_1 + y_1, \ldots, x_n + y_n) \\
    & = \bigl(\lambda(x_1 + y_1), \ldots, \lambda(x_n + y_n)\bigr) \\
    & = \bigl(\lambda x_1 + \lambda y_1, \ldots,
      \lambda x_n + \lambda y_n\bigr) \\
    & = (\lambda x_1, \ldots, \lambda x_n) +
      (\lambda y_1, \ldots, \lambda y_n) \\
    & = \lambda x + \lambda y
  \end{split}
\end{equation*}

% ******************************
\subsection{Question 16}
% ******************************

Show that $(a + b)x = ax + bx$ for all $a, b \in \mathbf{F}$ and all $x \in
\mathbf{F}^n$.

Solution:

Because $x \in \mathbf{F}^n$, according to the definition of $\mathbf{F}^n$, we
know that:

\[ x = (x_1, \ldots, x_n), x_j \in \mathbf{F}, j = 1, \ldots, n \]

Because $a, b, x_j \in \mathbf{F}$, their arithmetics satisfy the commutativity
and distributive property:

\begin{equation*}
  \begin{split}
    (a + b)x_j
    & = x_j (a + b) \\
    & = x_j a + x_j b \\
    & = a x_j + b x_j
  \end{split}
\end{equation*}

Therefore, according to the scalar multiplication on $\mathbf{F}^n$, we have:

\begin{equation*}
  \begin{split}
    (a + b)x
    & = (a + b)(x_1, \ldots, x_n) \\
    & = \bigl((a + b)x_1, \ldots, (a + b)x_n\bigr) \\
    & = (a x_1 + b x_1, \ldots, a x_n + b x_n) \\
    & = (a x_1, \ldots, a x_n) + (b x_1, \ldots, b x_n) \\
    & = ax + bx
  \end{split}
\end{equation*}

% =============================================================================
\section{EXERCISES 1.B}
% =============================================================================

% ******************************
\subsection{Question 1}
% ******************************

Prove that $-(-v) = v$ for every $v \in V$.

Solution:

Per 1.31, we know $-v = (-1)v$ for every $v \in V$. Therefore,

\[ -(-v) = (-1)(-v) = (-1)\bigl((-1)v\bigr) \]

Because $v$ is an element in the vector space $V$, that means the scalar
multiplication on $V$ satisfies the property of associativity, i.e., for $a, b
\in \mathbf{F}$ and $v \in V$, we have:

\[ (ab)v = a(bv)\]

and $1$ is the multiplicative identity, i.e.:

\[ 1v = v\]

Therefore,

\[ (-1)\bigl((-1)v\bigr) = \bigl((-1)(-1)\bigr)v = 1v = v \]

Therefore,

\[ -(-v) = v \]

% ******************************
\subsection{Question 2}
% ******************************

Suppose $a \in \mathbf{F}$, $v \in V$, and $av = 0$. Prove that $a = 0$ or
$v = 0$.

Solution:

We must be careful to not assume that $V$ is $\mathbf{F}^n$. Although we've
learned that $\mathbf{F}^n$ is an important vector space, vector space itself
can be a set of any kind of elements, not necessarily $\mathbf{F}^n$.

We also need to note that the $0$ on the right side is the zero vector in $V$,
not a complex number in $\mathbf{F}$.

We need to discuss this in two cases.

The first case is $a \neq 0$ where $0 \in \mathbf{F}$. Because $a \neq 0$,
$1/a$ exists and makes sense. Therefore, if we do scalar multiplication on the
left side, we have (applying the multiplicative associativity and the fact that
1 is the multiplicative identity):

\[ \frac{1}{a}(av) = (\frac{1}{a}a)v = 1v = v \]

And on the right side, according to 1.29, we have:

\[ \frac{1}{a}0 = 0 \]

Therefore, in this case,

\[ v = 0 \]

The second case is $v \neq 0$ where $0 \in V$. In this case:

\[ av = 0 = 0v \]

Because $v \neq 0$, $a$ must be $0$ to make the equation work. So $a = 0$ where
$0 \in \mathbf{F}$.

% ******************************
\subsection{Question 3}
% ******************************

Suppose $v, w \in V$. Explain why there exists a unique $x \in V$ such that
$v + 3x = w$.

Solution:

This can be explained as follows:

\begin{enumerate}
  \item The original expression is $v + 3x = w$.
  \item Applying the commutativity on the left side, we have $3x + v = w$.
  \item Adding the additive inverse on both sides, we have $3x + v + (-v) =
    w + (-v)$.
  \item Applying the associativity on the left side, we have $3x + \bigl(v +
    (-v)\bigr) = w + (-v)$.
  \item Because an element adding its additive inverse equals $0$, we have
    $3x + \bigl(v + (-v)\bigr) = 3x + 0 = w + (-v)$.
  \item Because an element adding the additive identity equals itself, we have
    $3x + 0 = 3x = w + (-v)$.
  \item Multiplying $1/3$ on both sides, we have $(\frac{1}{3})3x =
    (\frac{1}{3}3)x = 1x = x = \frac{1}{3}\bigl(w + (-v)\bigr)$.
  \item Because $w, (-v) \in V$ and $V$ is a vector space, so the addition and
    scalar multiplication are closed on $V$, so $\frac{1}{3}\bigl(w +
    (-v)\bigr) \in V$. So $x$ exists.
  \item Because $w, (-v)$ are specific elements in $V$, the arithmetics also
    produce the same result. So $x$ is unique.
\end{enumerate}

% ******************************
\subsection{Question 4}
% ******************************

The empty set is not a vector space. The empty set fails to satisfy only one of
the requirements listed in 1.19. Which one?

Solution:

The empty set doesn't satisfy the requirement of additive identity. A vector
space requires the existence of the additive identity, which means the vector
space must have at least one element. This conflicts with the fact that the
empty set doesn't have any elements.

% ******************************
\subsection{Question 5}
% ******************************

Show that in the definition of a vector space (1.19), the additive inverse
condition can be replaced with the condition that $0v = 0$ for all $v \in V$.
Here the $0$ on the left side is the number 0, and the $0$ on the right side is
the additive identity of $V$. (The phrase ``a condition can be replaced'' in a
definition means that the collection of objects satisfying the definition is
unchanged if the original condition is replaced with the new condition.)

Solution:

Because the number $0$ can be written as $\bigl(1 + (-1)\bigr)$, we have:

$0v = \bigl(1 + (-1)\bigr)v = 1v + (-1)v = v + (-1)v = 0$

Here we denote $(-1)v$ as $w$, so we have $v + w = 0$, which is the same as the
original condition of additive inverse.

% ******************************
\subsection{Question 6}
% ******************************

Let $\infty$ and $-\infty$ denote two distinct objects, neither of which is in
$\mathbf{R}$. Define an addition and scalar multiplication on $\mathbf{R} \cup
\{\infty\} \cup \{-\infty\}$ as you could guess from the notation. Specifically,
the sum and product of two real numbers is as usual, and for $t \in \mathbf{R}$
define:

\begin{itemize}
  \item
    \[
      t \infty = \begin{cases}
        -\infty, & if \ t < 0 \\
        0,       & if \ t = 0 \\
        \infty,  & if \ t > 0
      \end{cases}
    \]
  \item
    \[
      t (-\infty) = \begin{cases}
        \infty,   & if \ t < 0 \\
        0,        & if \ t = 0 \\
        -\infty,  & if \ t > 0
      \end{cases}
    \]
  \item $t + \infty = \infty + t = \infty$
  \item $t + (-\infty) = (-\infty) + t = -\infty$
  \item $\infty + \infty = \infty$
  \item $(-\infty) + (-\infty) = -\infty$
  \item $\infty + (-\infty) = 0$
\end{itemize}

Is $\mathbf{R} \cup \{\infty\} \cup \{-\infty\}$ a vector space over
$\mathbf{R}$? Explain

Solution:

We can solve this problem by examining $\mathbf{R} \cup \{\infty\} \cup
\{-\infty\}$ according to the definition of vector space.

To make it easier to write, let's use $\mathbb{R}$ to denote $\mathbf{R} \cup
\{\infty\} \cup \{-\infty\}$.

First of all, because $\mathbf{R}$ itself is already a vector space, we don't
need to verify that again, so we can focus on verifying whether the definition
still holds for $\infty$ and $-\infty$.

By examining the definition of addition and scalar multiplication above, we can
easily tell that \textbf{addition} and \textbf{scalar multiplication} are
closed on $\mathbb{R}$.

\textbf{Commutativity} also holds for all $e \in \mathbb{R}$.

For \textbf{associativity}, we need to examine the additive associativity and
multiplicative associativity separately. Let's examine the additive
associativity $(u + v) + w = u + (v + w)$ first. Again, we don't need to care
about the case when $u, v, w \in \mathbf{R}$. We need to examine when at least
one of $u, v, w$ is $\infty$ or $-\infty$. There are the following cases:

\begin{enumerate}
  \item $(\infty + v) + w$
  \item $(-\infty + v) + w$
  \item $(u + \infty) + w$
  \item $(u + -\infty) + w$
  \item $(u + v) + \infty$
  \item $(u + v) + -\infty$
  \item $(\infty + \infty) + w$
  \item $(-\infty + \infty) + w = 0 + w = w$, while $-\infty + (\infty + w) =
    -\infty + \infty = 0$ (not satisfied)
  \item $(\infty + -\infty) + w = 0 + w = w$, while $\infty + (-\infty + w) =
    \infty + -\infty = 0$ (not satisfied)
  \item $(-\infty + -\infty) + w$
  \item $(u + \infty) + \infty$
  \item $(u + -\infty) + \infty$ (not satisfied)
  \item $(u + \infty) + -\infty$ (not satisfied)
  \item $(u + -\infty) + -\infty$
  \item $(\infty + v) + \infty$
  \item $(-\infty + v) + \infty$
  \item $(\infty + v) + -\infty$
  \item $(-\infty + v) + -\infty$
  \item $(\infty + \infty) + \infty$
  \item $(-\infty + \infty) + \infty$ (not satisfied)
  \item $(\infty + -\infty) + \infty$
  \item $(\infty + \infty) + -\infty$ (not satisfied)
  \item $(-\infty + -\infty) + \infty$ (not satisfied)
  \item $(-\infty + \infty) + -\infty$
  \item $(\infty + -\infty) + -\infty$ (not satisfied)
  \item $(-\infty + -\infty) + -\infty$
\end{enumerate}

We can see that the additive associativity does not hold for $\mathbb{R}$, so
we can say $\mathbb{R}$ is not a vector space.

(TODO: Examine further to determine if the other properties hold.)

% =============================================================================
\section{EXERCISES 1.C}
% =============================================================================

% ******************************
\subsection{Question 1}
% ******************************

For each of the following subsets of $\mathbf{F}^3$, determine whether it is a
subspace of $\mathbf{F}^3$:

\begin{itemize}
  \item (a) \{$(x_1, x_2, x_3) \in \mathbf{F}^3: x_1 + 2x_2 + 3x_3 = 0$\};
  \item (b) \{$(x_1, x_2, x_3) \in \mathbf{F}^3: x_1 + 2x_2 + 3x_3 = 4$\};
  \item (c) \{$(x_1, x_2, x_3) \in \mathbf{F}^3: x_1 x_2 x_3 = 0$\};
  \item (d) \{$(x_1, x_2, x_3) \in \mathbf{F}^3: x_1 = 5x_3$\};
\end{itemize}

Solution:

Let's examine them one by one using 1.34, the conditions for a subspace.

% ------------------------------
\subsubsection*{(a)}
% ------------------------------

Let's examine \{$(x_1, x_2, x_3) \in \mathbf{F}^3: x_1 + 2x_2 + 3x_3 = 0$\}.
Let's call this subset $A$.

\begin{itemize}
  \item The additive identity $(0, 0, 0)$ is in $A$ because $0 + 2 \cdot 0 + 3
    \cdot 0 = 0$.
  \item $A$ is closed under addition because: For $u, w \in A$, according to
    the addition on $\mathbf{F}^3$, we have $u + w = (u_1, u_2, u_3) + (w_1,
    w_2, w_3) = (u_1 + w_1, u_2 + w_2, u_3 + w_3)$. We can also see that
    $(u_1 + w_1) + 2(u_2 + w_2) + 3(u_3 + w_3) = (u_1 + 2u_2 + 3u_3) + (w_1 +
    2w_2 + 3w_3) = 0 + 0 = 0$. Therefore, we know $u + w \in A$.
  \item $A$ is closed under scalar multiplication because: For $u \in A$ and
    $a \in \mathbf{F}$, according to the scalar multiplication on
    $\mathbf{F}^3$, $au = a(u_1, u_2, u_3) = (a u_1, a u_2, a u_3)$. We can
    also see that $(a u_1) + 2(a u_2) + 3(a u_3) = a(u_1 + 2u_2 + 3u_3) =
    a \cdot 0 = 0$. Therefore, we know that $au \in A$.
\end{itemize}

Therefore, the subset $A$ is a subspace of $\mathbf{F}^3$.

% ------------------------------
\subsubsection*{(b)}
% ------------------------------

Let's examine \{$(x_1, x_2, x_3) \in \mathbf{F}^3: x_1 + 2x_2 + 3x_3 = 4$\}.
Let's call this subset $B$.

\begin{itemize}
  \item The additive identity $(0, 0, 0)$ is \textbf{not} in $B$ because $0 +
    2 \cdot 0 + 3 \cdot 0 = 0 \neq 4$.
  \item $B$ is \textbf{not} closed under addition because: For $u, w \in B$,
    according to the addition on $\mathbf{F}^3$, we have $u + w = (u_1, u_2,
    u_3) + (w_1, w_2, w_3) = (u_1 + w_1, u_2 + w_2, u_3 + w_3)$. We can also
    see that $(u_1 + w_1) + 2(u_2 + w_2) + 3(u_3 + w_3) = (u_1 + 2u_2 + 3u_3) +
    (w_1 + 2w_2 + 3w_3) = 4 + 4 = 8 \neq 4$. Therefore, we know $u + w \notin
    B$.
  \item $B$ is \textbf{not} closed under scalar multiplication because: For $u
    \in B$ and $a \in \mathbf{F}$, according to the scalar multiplication on
    $\mathbf{F}^3$, $au = a(u_1, u_2, u_3) = (a u_1, a u_2, a u_3)$. We can
    also see that $(a u_1) + 2(a u_2) + 3(a u_3) = a(u_1 + 2u_2 + 3u_3) =
    a \cdot 0 = 0 \neq 4$. Therefore, we know that $au \notin B$.
\end{itemize}

Therefore, the subset $B$ is \textbf{not} a subspace of $\mathbf{F}^3$.

% ------------------------------
\subsubsection*{(c)}
% ------------------------------

Let's examine \{$(x_1, x_2, x_3) \in \mathbf{F}^3: x_1 x_2 x_3 = 0$\}. Let's
call this subset $C$. (Note the $C$ in this section is not the set of the
complex numbers.)

\begin{itemize}
  \item The additive identity $(0, 0, 0)$ is in $C$ because $0 \cdot 0 \cdot 0
    = 0$.
  \item $C$ is \textbf{not} closed under addition because: For $u, w \in C$,
    according to the addition on $\mathbf{F}^3$, we have:
    \begin{equation*}
      \begin{split}
        u + w
        & = (u_1, u_2, u_3) + (w_1, w_2, w_3) \\
        & = (u_1 + w_1, u_2 + w_2, u_3 + w_3)
      \end{split}
    \end{equation*}
    We can also see that:
    \begin{equation*}
      \begin{split}
        & (u_1 + w_1) \cdot (u_2 + w_2) \cdot (u_3 + w_3) \\
        & = u_1 u_2 u_3 + u_1 u_2 w_3 + u_1 w_2 u_3 + u_1 w_2 w_3 +
          w_1 u_2 w_3 + w_1 u_2 w_3 + w_1 w_2 u_3 + w_1 w_2 w_3 \\
        & = 0 + u_1 u_2 w_3 + u_1 w_2 u_3 + u_1 w_2 w_3 + w_1 u_2 w_3 +
          w_1 u_2 w_3 + w_1 w_2 u_3 + 0 \\
        & = u_1 u_2 w_3 + u_1 w_2 u_3 + u_1 w_2 w_3 + w_1 u_2 w_3 + w_1 u_2 w_3
          + w_1 w_2 u_3
      \end{split}
    \end{equation*}
    Because $u_1 u_2 u_3 = 0$ and $w_1 w_2 w_3 = 0$, we know at least one of
    $u_1, u_2, u_3$ is 0 and at least one of $w_1, w_2, w_3$ is 0. Among all
    the cases, in some cases the expression above is not equal to 0. For
    example, when $u_1, u_2 \neq 0, u_3 = 0$ and $w_1 = 0, w_2, w_3 \neq 0$,
    the expression above is $u_1 u_2 w_3 + 0 + u_1 w_2 w_3 + 0 + 0 + 0 \neq 0$.
    Therefore, $C$ is not closed under addition.
  \item $C$ is closed under scalar multiplication because: For $u \in C$ and
    $a \in \mathbf{F}$, according to the scalar multiplication on
    $\mathbf{F}^3$, $au = a(u_1, u_2, u_3) = (a u_1, a u_2, a u_3)$. We can
    also see that $(a u_1) \cdot (a u_2) \cdot (a u_3) = a^3(u_1 u_2 u_3) =
    a^3 \cdot 0 = 0$. Therefore, we know that $au \in C$.
\end{itemize}

Therefore, the subset $C$ is \textbf{not} a subspace of $\mathbf{F}^3$.

% ------------------------------
\subsubsection*{(d)}
% ------------------------------

Let's examine \{$(x_1, x_2, x_3) \in \mathbf{F}^3: x_1 = 5x_3$\}.
Let's call this subset $D$.

\begin{itemize}
  \item The additive identity $(0, 0, 0)$ is in $D$ because $0 = 5 \cdot 0$.
  \item $D$ is closed under addition because: For $u, w \in D$, according to
    the addition on $\mathbf{F}^3$, we have $u + w = (u_1, u_2, u_3) + (w_1,
    w_2, w_3) = (u_1 + w_1, u_2 + w_2, u_3 + w_3)$. We can also see that $(u_1
    + w_1) = (5u_3 + 5w_3) = 5(u_3 + w_3)$. Therefore, we know $u + w \in D$.
  \item $D$ is closed under scalar multiplication because: For $u \in D$ and $a
    \in \mathbf{F}$, according to the scalar multiplication on $\mathbf{F}^3$,
    $au = a(u_1, u_2, u_3) = (a u_1, a u_2, a u_3)$. We can also see that
    $a u_1 = a (5u_3) = 5(a u_3)$. Therefore, we know that $au \in D$.
\end{itemize}

Therefore, the subset $D$ is a subspace of $\mathbf{F}^3$.

% ******************************
\subsection{Question 2}
% ******************************

Verify all the assertions in Example 1.35:

\begin{itemize}
  \item (a) If $b \in \mathbf{F}$, then \{ $(x_1, x_2, x_3, x_4) \in
    \mathbf{F}^4: x_3 = 5x_4 + b$ \} is a subspace of $\mathbf{F}^4$ if and
    only if $b=0$.
  \item (b) The set of continuous real-valued functions on the interval [0, 1]
    is a subspace of $\mathbf{R}^{[0,1]}$.
  \item (c) The set of differentiable real-valued functions on $\mathbf{R}$ is
    a subspace of $\mathbf{R}^{\mathbf{R}}$.
  \item (d) The set of differentiable real-valued functions $f$ on the interval
    $(0, 3)$ such that $f'(2) = b$ is a subspace of $\mathbf{R}^{(0,3)}$ if and
    only if $b = 0$.
  \item (e) The set of all sequences of complex numbers with limit 0 is a
    subspace of $\mathbf{C}^{\infty}$.
\end{itemize}

Solution:

% ------------------------------
\subsubsection*{(a)}
% ------------------------------

Let's call this set $A$ and examine the three conditions that make a set a
subspace.

Let's examine the direction of sufficiency : if $b = 0$, then $A$ is a subspace
of $\mathbf{F}^4$. When $b = 0$, set $A$ is \{ $(x_1, x_2, x_3, x_4) \in
\mathbf{F}^4: x_3 = 5x_4 + 0 = 5x_4$ \}. Therefore:

\begin{itemize}
  \item The additive identity $(0, 0, 0, 0)$ is in $A$ because $0 = 5 \cdot 0$.
  \item $A$ is closed under addition because: For $u, w \in A$, according to
    the addition on $\mathbf{F}^4$, we have $u + w = (u_1, u_2, u_3, u_4) +
    (w_1, w_2, w_3, w_4) = (u_1 + w_1, u_2 + w_2, u_3 + w_3, u_4 + w_4)$. We
    can also see that $(u_3 + w_3) = (5u_4 + 5w_4) = 5(u_4 + w_4)$. Therefore,
    we know $u + w \in A$.
  \item $A$ is closed under scalar multiplication because: For $u \in A$ and $a
    \in \mathbf{F}$, according to the scalar multiplication on $\mathbf{F}^4$,
    $au = a(u_1, u_2, u_3, u_4) = (a u_1, a u_2, a u_3, a u_4)$. We can also
    see that $a u_3 = a (5u_4) = 5(a u_4)$. Therefore, we know that $au \in A$.
\end{itemize}

Therefore, we know that when $b = 0$, $A$ is a subspace of $\mathbf{F}^4$.

Now let's examine the direction of necessity: if $A$ is a subspace of
$\mathbf{F}^4$, then $b = 0$. Because now we know $A$ is a subspace, the
additive identity $(0, 0, 0, 0)$ must be in $A$. Therefore, the third element
$0$ must be equal to $5 \cdot 0 + b$ where ``$0$'' is the fourth element. So
$b = 0$.

% ------------------------------
\subsubsection*{(b)}
% ------------------------------

Let's call this set $B$. To solve this question, we need to review the
definition of the notation $\mathbf{F}^S$. See section 1.23 and section 1.24 in
the textbook. In this question, $\mathbf{R}^{[0,1]}$ refers to the set of
functions from $[0,1]$ to $\mathbf{R}$. Note the functions in
$\mathbf{R}^{[0,1]}$ don't have to be continuous.

For two functions $f, g \in \mathbf{R}^{[0,1]}$, the sum $f + g$ is defined by
$(f + g)(x) = f(x) + g(x)$ for all $x \in [0,1]$. (Because $f(x) \in
\mathbf{R}$ and $g(x) \in \mathbf{R}$, $f(x) + g(x) \in \mathbf{R}$, we know
$(f + g)(x)$ is also a function from $[0,1]$ to $\mathbf{R}$, so $(f + g)(x)
\in \mathbf{R}^{[0,1]}$.)

For $\lambda \in \mathbf{R}$ and $f \in \mathbf{R}^{[0,1]}$, the product
$\lambda f$ is defined by $(\lambda f)(x) = \lambda f(x)$ for all $x \in [0,1]$.
(Similarly, we can verify that $(\lambda f)(x) \in \mathbf{R}^{[0,1]}$.)

The additive identity of $\mathbf{R}^{[0,1]}$ is the function $0(x) = 0$ for
all $x \in [0,1]$.

Using the knowledge of real analysis, we can tell that:
\begin{itemize}
  \item $0(x) = 0$ is a continuous real-valued function so it belongs to $B$.
  \item Adding two continuous real-valued functions $f + g$ produces another
    continuous real-valued functions, so $f + g \in B$, so $B$ is closed under
    addition.
  \item Multiplying a continuous real-valued function with a real number also
    produces a continuous real-valued function, therefore, $\lambda f \in B$,
    so $B$ is closed under scalar multiplication.
\end{itemize}

Therefore, $B$ is a subspace of $\mathbf{R}^{[0,1]}$.

% ------------------------------
\subsubsection*{(c)}
% ------------------------------

TODO: I need to come back after learning mathematical analysis.

% ------------------------------
\subsubsection*{(d)}
% ------------------------------

TODO: I need to come back after learning mathematical analysis.

% ------------------------------
\subsubsection*{(e)}
% ------------------------------

TODO: I need to come back after learning mathematical analysis.

% ******************************
\subsection{Question 3}
% ******************************

TODO: I need to come back after learning mathematical analysis.

% ******************************
\subsection{Question 4}
% ******************************

TODO: I need to come back after learning mathematical analysis.

% ******************************
\subsection{Question 5}
% ******************************

Is $\mathbf{R}^2$ a subspace of the complex vector space $\mathbf{C}^2$?

Solution:

First of all, we need to figure out what $\mathbf{R}^2$ and $\mathbf{C}^2$
refer to exactly. Based on the definition 1.10 for $\mathbf{F}^n$, we can see
that

\[
  \mathbf{R}^2 = \{(x_1, x_2): x_1, x_2 \in \mathbf{R} \}
\]

and

\[
  \mathbf{C}^2 = \{(x_1, x_2): x_1, x_2 \in \mathbf{C} \}
\]

We also need to understand how addition and scalar multiplication are defined
on them. This can be found in definition 1.13 and definition 1.14.

Now let's examine if $\mathbf{R}^2$ meets the three conditions that make it a
subspace.

\begin{enumerate}
  \item The additive identity on $\mathbf{C}^2$ is $(0 + 0i, 0 + 0i) = (0, 0)$.
    For $\forall (x_1, x_2) \in \mathbf{R}^2$, $(0, 0) + (x_1, x_2) = (0 + x_1,
    0 + x_2) = (x_1, x_2)$. So the additive identity is in $\mathbf{R}^2$.
  \item $\mathbf{R}^2$ is closed under addition because: For $u, w \in
    \mathbf{R}^2$, we have $u + w = (u_1, u_2) + (w_1, w_2) = (u_1 + w_1, u_2 +
    w_2).$ Because $u_1, u_2, w_1, w_2 \in \mathbf{R}$, $u_1 + w_1 \in
    \mathbf{R}$ and $u_2 + w_2 \in \mathbf{R}$, so $(u_1 + w_1, u_2 + w_2) \in
    \mathbf{R}^2$, so $u + w \in \mathbf{R}^2$, so $\mathbf{R}^2$ is closed
    under addition.
  \item $\mathbf{R}^2$ is closed under scalar multiplication because: For $u
    \in \mathbf{R}^2$ and $a \in \mathbf{R}$, $au = a(u_1, u_2) = (a u_1,
    a u_2)$. Because $u_1, u_2, a \in \mathbf{R}$, $a u_1 \in \mathbf{R}$ and
    $a u_2 \in \mathbf{R}$, so $(a u_1, a u_2) \in \mathbf{R}^2$, so
    $\mathbf{R}^2$ is closed under scalar multiplication.
\end{enumerate}

Therefore, $\mathbf{R}^2$ is a subspace of $\mathbf{C}^2$.

% ******************************
\subsection{Question 6}
% ******************************

TODO

% ******************************
\subsection{Question 7}
% ******************************

TODO

% ******************************
\subsection{Question 8}
% ******************************

TODO

% ******************************
\subsection{Question 9}
% ******************************

TODO

% ******************************
\subsection{Question 10}
% ******************************

TODO

% ******************************
\subsection{Question 11}
% ******************************

TODO

% ******************************
\subsection{Question 12}
% ******************************

TODO

% ******************************
\subsection{Question 13}
% ******************************

TODO

% ******************************
\subsection{Question 14}
% ******************************

TODO

% ******************************
\subsection{Question 15}
% ******************************

TODO

% ******************************
\subsection{Question 16}
% ******************************

TODO

% ******************************
\subsection{Question 17}
% ******************************

TODO

% ******************************
\subsection{Question 18}
% ******************************

TODO

% ******************************
\subsection{Question 19}
% ******************************

TODO

% ******************************
\subsection{Question 20}
% ******************************

TODO

% ******************************
\subsection{Question 21}
% ******************************

TODO

% ******************************
\subsection{Question 22}
% ******************************

TODO

% ******************************
\subsection{Question 23}
% ******************************

TODO

% ******************************
\subsection{Question 24}
% ******************************

TODO

\end{document}
