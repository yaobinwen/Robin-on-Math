\documentclass[12pt, letterpaper, oneside]{book}
\usepackage{pseudocode}
\usepackage{amsfonts}
\usepackage{amsmath}
\usepackage{amssymb}
\usepackage{csquotes}
\usepackage{float}
\usepackage{hyperref}
\usepackage[letterpaper, textwidth=7.5in, textheight=8in]{geometry}
\hypersetup{
  colorlinks=true,
  linkcolor=blue,
  filecolor=magenta,
  urlcolor=blue,
}
\usepackage{titlesec}
\usepackage{parskip}
\usepackage{xcolor}

\setcounter{secnumdepth}{4}

\title{My commonly-used first-order logic predicates and functions}
\author{Yaobin Wen}
\date{Jan 2024}

\begin{document}

\maketitle
\tableofcontents

\chapter*{Overview}
\addcontentsline{toc}{chapter}{Overview}

I find first-order logic calculus is a strict and clear way to make statements. If used appropriately, it can eliminate
ambiguity and confusion in the text.

However, the first-order logic's built-in predicates and functions are often insufficient to express complex statements.
Therefore, I define the additional predicates and functions in this document that I will use in other documents.

% =============================================================================
%
\chapter{Predicates}
%
% =============================================================================

\begin{itemize}
  \item $AssertTrue(pred)$: Assert the given predicate is true.
  \item $ExactlyOneIsTrue(pred1, pred2, \ldots, predN)$: Exactly one of the predicates of \{ pred1, pred2, $\ldots$,
        predN \} is true.
  \item $HasMaxElement(S)$: $S \ne \emptyset \land IsOrdered(S) \land sup S \in S$.
  \item $HasMinElement(S)$: $S \ne \emptyset \land IsOrdered(S) \land inf S \in S$.
  \item $InheritsOrder(Q, P)$: $IsOrdered(P) \land Q \subset P$
  \item $IsAntichain(P)$: The ordered set $P$ is an antichain.
  \item $IsBijection(f, A, B)$: Function $f$ is a bijection from the set $A$ to set $B$.
  \item $IsBoundedAbove(E, S)$: Subset $E$ is bounded above in the set $S$.
  \item $IsBoundedBelow(E, S)$: Subset $E$ is bounded below in the set $S$.
  \item $IsBounded(E, S)$: $IsBoundedAbove(E, S) \land IsBoundedBelow(E, S)$
  \item $IsChain(P)$: The ordered set $P$ is a chain.
  \item $IsEven(i)$: Integer $i$ is an even number (e.g., $-880$, $0$, $1938$).
  \item $IsFinite(S)$: Set $S$ is finite.
  \item $IsInfinite(S)$: Set $S$ is infinite.
  \item $IsLinearlyOrdered(P)$: $IsChain(P)$.
  \item $IsLowerBound(e, E, S)$: Element $e$ is an lower bound of $E$ in $S$ where $E \subset S$.
  \item $IsOdd(i)$: Integer $i$ is an odd number (e.g., $-79$, $3$).
  \item $IsOrdered(S)$: $S$ is an ordered set.
  \item $IsTotallyOrdered(P)$: $IsChain(P)$.
  \item $IsUpperBound(e, E, S)$: Element $e$ is an upper bound of $E$ in $S$ where $E \subset S$.
\end{itemize}

% =============================================================================
%
\chapter{Functions}
%
% =============================================================================

\begin{itemize}
  \item $MaxElement(S) \rightarrow e$: $HasMaxElement(S) \land e = sup S$
  \item $MinElement(S) \rightarrow e$: $HasMinElement(S) \land e = inf S$
  \item $PickElement(S) \rightarrow e$: Pick an element from the set $S$ and return it as $e$.
\end{itemize}

\end{document}
