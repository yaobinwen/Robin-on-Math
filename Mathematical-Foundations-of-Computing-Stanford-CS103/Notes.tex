% Regarding `oneside` (https://stackoverflow.com/a/8371473/630364):
%
% `oneside` removes the blank pages between chapters.
% "Note that this method make the margins of all the pages the same. In
% `twoside`, the margins are different for the odd and the even pages".
\documentclass[12pt, letterpaper, oneside]{book}

\usepackage{amsfonts}
\usepackage{amsmath}
\usepackage{amssymb}
\usepackage{csquotes}
\usepackage{float}
\usepackage[letterpaper, textwidth=7.5in, textheight=8in]{geometry}
\usepackage{hyperref}
\usepackage{parskip}
\usepackage{pseudocode}
\usepackage{titlesec}
\usepackage{xcolor}

\hypersetup{
  colorlinks=true,
  linkcolor=blue,
  filecolor=magenta,
  urlcolor=blue,
}

\setcounter{secnumdepth}{4}

\title{
  Notes on \textit{Stanford: CS103: Mathematical Foundations of Computing}
}
\author{Yaobin Wen}
\date{July 2023}

\begin{document}

\maketitle
\tableofcontents

\chapter*{Overview}
\addcontentsline{toc}{chapter}{Overview}

This document contains my study notes of the Stanford course \textit{CS103:
Mathematical Foundations of Computing}. I use it for a few purposes:

\begin{enumerate}
  \item As a reference to quickly refresh my memory on the subjects.
  \item Keep the notes to help me understand the text that is not obvious for
    me to comprehend.
\end{enumerate}

% =============================================================================
%
% Set Theory
%
% =============================================================================

\chapter{Set Theory}

% =============================================================================
\section{Is 0 a natural number?}
% =============================================================================

Is $0$ a natural number? Does $\mathbb{N}$ include $0$?

There is no ``yes'' or ``no'' answer to these questions. Take a look at
\href{https://math.stackexchange.com/q/283/665777}{this question} and you will
find people all over the world have all kinds of understanding such as:
\begin{itemize}
  \item $0$ is definitely \textbf{NOT} a natural number and does not belong to
    $\mathbb{N}$.
  \item There are two conventions: In one convention, $0$ is not a natural
    number; in another convention, $0$ is a natural number.
  \item Natural numbers include 0; those positive integers are called ``whole
    numbers'':
    \begin{displayquote}
      I see plenty of both these days, but when I was at school and at
      university, I almost only saw them defined to be {0, 1, ..}. The elements
      of {1, 2, ..} were called the whole numbers in my school days.
    \end{displayquote}
  \item Natural numbers don't include 0; whole numbers, denoted as $\mathbb{W}$,
    include $0$.
  \item $\mathbb{N}$ includes 0; $\mathbb{N^+}$ is all the positive integers so
    it doesn't include 0.
  \item yadda yadda yadda...
\end{itemize}

So I don't think it makes sense to argue whether $0$ is a natural number or not.
We just need to define it clearly and move on.

In CS103, $\mathbb{N}$ represents the natural numbers that \textbf{include} $0$.
But if you read 18.100A, you'll see in that course, $\mathbb{N}$ does not
include $0$.

That's fine. The world is still in peace. So in this note, I will respect the
choice of CS103 and treat $\mathbb{N}$ as the set of all the non-negative
integers, i.e., \[\{0, 1, 2, 3, \ldots\}\].

% =============================================================================
\section{Introduction to Set Theory}
% =============================================================================

This first lecture introduces set theory. Because I have learned them in MIT
OCW 18.100A Real Analysis, I won't repeat the notes again because they can be
found in the notes for that course. Here, I will just mention a few points that
are not covered in 18.100A.

\begin{itemize}
  \item Set difference: 18.100A uses $A \setminus B$ to denote ``A minus B''.
    CS103 says $A - B$ is also used.
  \item \textbf{Symmetric difference}: $A \bigtriangleup B = (A \setminus B)
    \cup (B \setminus A) = (A \cup B) \setminus (A \cap B)$. For example, if
    $A = \{1, 2, 3\}$ and $B = \{3, 4, 5\}$, then $A \bigtriangleup B = \{1, 2,
    4, 5\}$. \colorbox{red}{\textcolor{yellow}{TODO:}} Add a Venn diagram.
\end{itemize}

% =============================================================================
\section{Cardinality}
% =============================================================================

% ******************************
\subsection{$|\mathbb{N}|$ and $|\mathbb{Z}|$}
% ******************************

I've already learned that $|\mathbb{Z}|$ and the set of all the even (or odd)
numbers have the same cardinality, but in CS103 I just learned $|\mathbb{N}| =
|\mathbb{Z}|$. The bijection (see 18.100A) between $\mathbb{N}$ and $\mathbb{Z}$
is created in the following way:
\begin{itemize}
  \item All the non-negative even numbers (including $0$) in $\mathbb{N}$ are
    matched to one non-negative integers in $\mathbb{Z}$.
  \item All the non-negative odd numbers in $\mathbb{N}$ are matched to one
    negative integers in $\mathbb{Z}$.
\end{itemize}

Visualized, this can be shown as:
\begin{itemize}
  \item $0 \longleftrightarrow 0$
  \item $1 \longleftrightarrow -1$
  \item $2 \longleftrightarrow 1$
  \item $3 \longleftrightarrow -2$
  \item $4 \longleftrightarrow 2$
  \item $5 \longleftrightarrow -3$
  \item $6 \longleftrightarrow 3$
  \item $7 \longleftrightarrow -4$
  \item $8 \longleftrightarrow 4$
  \item etc.
\end{itemize}

In general, the bijection $f: \mathbb{N} \rightarrow \mathbb{Z}$ is defined as:

\[
  f(n) = \begin{cases}
    n / 2,              & if \ MOD(n) = 0 \\
    (-n - 1) / 2,       & if \ MOD(n) = 1
  \end{cases}
\]

where $n \in \mathbb{N}$.

% ******************************
\subsection{Cantor's diagonalization proof}
% ******************************

\href{https://en.wikipedia.org/wiki/Cantor%27s_diagonal_argument}{Cantor's
diagonalization proof} is used to prove $|S| < |\mathcal{P}(S)|$.

% ******************************
\subsection{$|Programs| < |Problems|$}
% ******************************

This is the most interesting conclusion I've seen in this lecture. The proof
process can be summarized as follows:
\begin{itemize}
  \item Because every valid computer program is essentially a string, but not
    every string is a valid program, we can see \[|Program| \leq |Strings|\].
  \item The problems in the world may or may not deal with sets of strings.
    Suppose $S$ denotes a set of strings, so one kind of problems is: Given a
    string $s$, determine whether $s \in S$. For example:
    \begin{itemize}
      \item Suppose S = \{ ``a'', ``b'', ``c'', $\ldots$, ``z'' \}, then the
        problem can be: Given a string $s$, determine whether $s$ is an English
        letter.
        \item Suppose S = \{ ``0'', ``1'', ``2'', ``3'', $\ldots$ \}, then the
        problem can be: Given a string $s$, determine whether $s$ represents a
        natural number.
    \end{itemize}
  \item Not to mention that other kinds of problems can exist and may not be
    converted to problems that deal with sets of strings.
  \item The conclusion is: $|Sets\ of\ strings| \leq |Problems|$. They could be
    equal, if all the problems can be converted to problems that deal with sets
    of strings.
  \item Therefore:
    \[|Program| \leq |Strings| < |Sets\ of\ strings| \leq |Problems|\]
\end{itemize}

% =============================================================================
%
% Mathematical Proofs
%
% =============================================================================

\chapter{Mathematical Proofs}

\begin{enumerate}
  \item $\blacksquare$ means ``end of proof''.
  \item Use the ``mugga mugga'' test to verify if the proof is written in valid
    sentences: Replace all the mathematical notation with ``mugga mugga''. What
    comes back should still be a valid sentence.
  \item The section ``Proofs as a Dialog'' is interesting: It says the variables
    used in a proof can be divided into three categories:
    \begin{enumerate}
      \item Proof writer picks. Example: ``Let r = n + 1''
      \item Proof reader picks. Example: ``Consider some $n \in \mathbb{N}$''
      \item Neither picks: The variable's value is determined by some laws or
        definition. Example: ``If $n$ is even, there exists a $k$ so that $n =
        2k$''. In this case, if the reader chooses an arbitrary even number $n$,
        the corresponding $k$'s value is not chosen by either the reader or the
        writer but by the definition of ``even''.
    \end{enumerate}
    Knowing which variables are picked by whom can help verify if the proof
    makes sense. Don't change the variables you (the proof writer) don't own.
\end{enumerate}

% =============================================================================
%
% Indirect Proofs
%
% =============================================================================

\chapter{Indirect Proofs}

(TODO)

% =============================================================================
%
% Propositional Logic
%
% =============================================================================

\chapter{Propositional Logic}

% =============================================================================
\section{Proposition}
% =============================================================================

% ******************************
\subsection{Definition}
% ******************************

A \textbf{proposition} is a statement that is, by itself, either \textbf{true}
or \textbf{false}.
\begin{itemize}
  \item Commands are not propositions. Example: Open the door.
  \item Questions are not propositions. Example: What day is it today?
\end{itemize}

A \textbf{propositional variable}, usually a lower-case English letter such as
$p$, $q$, $r$, represents a proposition.

A \textbf{propositional connective} expresses how propositions are related.
Some of the connectives are:
\begin{itemize}
  \item $\lnot p$: ``NOT p''; logical negation.
  \item $p \land q$: ``p AND q''; logical conjunction.
  \item $p \lor q$: ``p OR q''; logical disjunction (inclusive). ``inclusive''
    means $p$ and $q$ can be $true$ at the same time, while ``exclusive or''
    means $p$ and $q$ cannot be $true$ at the same time.
  \item $p \rightarrow q$: ``p implies q''; material condition.
  \item $p \leftrightarrow q$: ``p if and only if q'', which also means ``(p
    implies q) AND (q implies p)''.
  \item $\top$: ``(always) true'' ($\top$ looks like a upper-case ``T'' that
    can represent ``True''.)
  \item $\bot$: ``(always) false''
\end{itemize}

% ******************************
\subsection{Example of using $\top$ and $\bot$}
% ******************************

$\bot$ can be used to describe how proof by contradiction works. Suppose that
you want to prove $p$ is $true$ using proof by contradiction. The usual steps
are as follows:
\begin{enumerate}
  \item Assume $p$ is $false$.
  \item Derive a conclusion that is known as $false$ (e.g., ``3 is even'').
  \item Conclude that $p$ is $true$.
\end{enumerate}

Described in propositional logic, it is
\[
  (\lnot p \rightarrow \bot) \rightarrow p
\].

% ******************************
\subsection{Logical operator precedence}
% ******************************

All the logical operators are \textbf{right-associative}.

In the order of highest to lowest, they are:
\begin{enumerate}
  \item $\lnot$
  \item $\land$
  \item $\lor$
  \item $\rightarrow$
  \item $\leftrightarrow$
\end{enumerate}

For example, the statement
\[
  \lnot x \rightarrow y \lor z \rightarrow x \lor y \and z
\]
can be grouped as follows:
\[
  (\lnot x) \rightarrow ((y \lor z) \rightarrow (x \lor (y \and z)))
\]

% ******************************
\subsection{Truth table}
% ******************************

I have already learned the truth tables for $\lnot$, $\land$, and $\lor$. What
is surprising to me is the truth table for $\rightarrow$:

\begin{table}[H]
  \centering
  \begin{tabular}{|c|c|c|ll}
  \cline{1-3}
  p & q & $p \rightarrow q$ &  &  \\ [1ex] \cline{1-3}
  F & F & T                 &  &  \\ [0.5ex] \cline{1-3}
  F & T & T                 &  &  \\ [0.5ex] \cline{1-3}
  T & F & F                 &  &  \\ [0.5ex] \cline{1-3}
  T & T & T                 &  &  \\ [0.5ex] \cline{1-3}
  \end{tabular}
  \caption{Truth table for $p \rightarrow q$}
\end{table}

The first two lines are the most confusing to many people, and there are many
questions about them. Here are some of them:
\begin{itemize}
  \item \href{https://philosophy.stackexchange.com/q/26719/44172}{Shouldn't statements be considered equivalent based on their meaning rather than truth tables?}
  \item \href{https://philosophy.stackexchange.com/q/34082/44172}{Why are conditionals with false antecedents considered true?}
  \item \href{https://math.stackexchange.com/q/3098664/665777}{How Implication or Material/Concrete Conditional works when the antecedent is false and the consequent is true}
  \item \href{https://math.stackexchange.com/q/70736/665777}{In classical logic, why is ($p \rightarrow q$) True if p is False and q is True?}
  \item And many more...
\end{itemize}

I read some of the posts and then decided to stop because this looks like a
deep rabbit hole. For now, it is better to just accept the truth table and move
on. But to help understand them a little bit:
\begin{itemize}
  \item $p \rightarrow q \equiv \lnot p \lor q \equiv \lnot (p \land \lnot q)$.
  \item $\lnot (p \rightarrow q) \equiv p \land \lnot q$.
  \item It's helpful to think about ``Ex falso sequitur quodlibet'' which means
    ``from what is false any assertion validly follows''.
\end{itemize}

\colorbox{red}{\textcolor{yellow}{TODO:}} Figure out why $F \rightarrow F$ is
$T$ and $F \rightarrow T$ is $T$. Or figure out why $p \rightarrow q$ is
equivalent to $\lnot p \lor q$.

Here is the truth table for $p \leftrightarrow q$:

\begin{table}[H]
  \centering
  \begin{tabular}{|c|c|c|ll}
  \cline{1-3}
  p & q & $p \leftrightarrow q$ &  &  \\ [1ex] \cline{1-3}
  F & F & T                     &  &  \\ [0.5ex] \cline{1-3}
  F & T & F                     &  &  \\ [0.5ex] \cline{1-3}
  T & F & F                     &  &  \\ [0.5ex] \cline{1-3}
  T & T & T                     &  &  \\ [0.5ex] \cline{1-3}
  \end{tabular}
  \caption{Truth table for $p \rightarrow q$}
\end{table}

% ------------------------------
\subsubsection{Vacuously true; trivially true}
% ------------------------------

An implication with a false antecedent is called \textbf{vacuously true}, such
as $F \rightarrow F: T$ and $F \rightarrow T: T$.

An implication with a true consequent is called \textbf{trivially true}, such
as $F \rightarrow T: T$ and $T \rightarrow T: T$.

Note that the implication $F \rightarrow F: T$ is \textbf{both vacuous and
trivial}.

Note that the implication $T \rightarrow F: F$ is \textbf{neither of them}.

% ******************************
\subsection{de Morgan's Laws}
% ******************************

\begin{itemize}
  \item $\lnot (p \land q) \equiv \lnot p \lor \lnot q$
  \item $\lnot (p \lor q) \equiv \lnot p \land \lnot q$
\end{itemize}

\chapter*{References}
\addcontentsline{toc}{chapter}{References}

\begin{itemize}
  \item $[1]$ \href{https://web.stanford.edu/class/cs103/}{CS103: Mathematical Foundations of Computing}
\end{itemize}

\end{document}
