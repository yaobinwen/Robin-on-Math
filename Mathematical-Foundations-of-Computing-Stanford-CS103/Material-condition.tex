\documentclass[12pt, letterpaper]{article}

\usepackage{amssymb}
\usepackage{csquotes}
\usepackage{float}
\usepackage{hyperref}
\usepackage{parskip}
\usepackage{xcolor}

% Configure the display style of hyperlinks.
\hypersetup{
  colorlinks=true,
  linkcolor=blue,
  filecolor=magenta,
  urlcolor=blue,
}

\title{
  My Understanding of Relation Between Material Condition and Logical
  Implication
}
\author{
  Yaobin Wen\thanks{Funded by my own humility, interest, and dedication.}
}
\date{July 2023}

\begin{document}
\maketitle

\begin{abstract}
  That using the material condition (i.e., one of the many binary truth
  functions) to represent the logical implication has caused a lot of confusion.
  One can search questions such as \textit{why is $p \rightarrow q$ $True$ if
  $p$ is $False$ and $q$ is $True$?}, and many similar questions would show up.
  I myself was confused, too. My confusion was resolved (so far) after reading
  the related sections in the article \textit{Propositional Logic} that was
  published by \textit{Stanford Encyclopedia of Philosophy}\footnote{I will
  refer to this article as \textit{SEP-PL}}. This article has my notes of
  learning.

  In general, \textit{SEP-PL}, although admitted that there is controversy in
  this topic, favors the position that it is wrong to treat material condition
  as a way to describe logical implication (``This truth function is often
  described, controversially, as giving the truth conditions..., or even,
  erroneously, for the relation `implies'. ''). This is the key to dissolve the
  confusion. Actually, \textit{SEP-PL} provides a bigger picture for us to see
  how propositional logic is built, especially how the material condition came
  into being in the first place, which helped us to see why it is inappropriate
  to use it to indicate ``implication''.

  This article is divided into the following sections. The first section
  presents the confusion that many people have asked online. This confusion was
  also my motivation to study the topic of this article. The next section
  provides a quick answer to the question, which is essentially a summary of
  the last section that explains answer in detail.
\end{abstract}

% =============================================================================
\section{The Confusion}
% =============================================================================

When studying study propositional logic, you typically start with the
definitions of ``proposition'' (which is usually represented by a lower-case
letter such as ``p'') and ``connective'' (such as $\lnot$, $\land$, and $\lor$),
followed by the definition of ``truth table''. The truth tables for $\lnot$
(logical NOT), $\land$ (logical AND), and $\lor$ (logical OR) are not difficult
to grasp because they match our daily experience. For example:

\begin{itemize}
  \item $\lnot True$ = $False$. Example: NOT it is raining = It is not raining.
  \item Alice is here $\land$ Bob is here = It is true that both Alice and Bob
    are here.
  \item Alice will attend the meeting $\lor$ Bob will attend the meeting =
    Either Alice or Bob (or both of them) will attend the meeting. (Note that
    $\lor$ is \textit{inclusive or}, so the case of ``both'' can happen.)
\end{itemize}

However, depending on the textbooks, instructors, or courses, you may also be
taught that $\rightarrow$ means ``implication'' and $p \rightarrow q$ means ``p
implies q''. Its truth table is said to be as follows:

\begin{table}[H]
  \centering
  \begin{tabular}{|c|c|c|ll}
  \cline{1-3}
  p & q & $p \rightarrow q$ &  &  \\ [1ex] \cline{1-3}
  F & F & T                 &  &  \\ [0.5ex] \cline{1-3}
  F & T & T                 &  &  \\ [0.5ex] \cline{1-3}
  T & F & F                 &  &  \\ [0.5ex] \cline{1-3}
  T & T & T                 &  &  \\ [0.5ex] \cline{1-3}
  \end{tabular}
  \caption{Truth table for $p \rightarrow q$}
\end{table}

You may quickly raise questions about the first two lines:
\begin{itemize}
  \item Why is $p \rightarrow q$ $True$ if both $p$ and $q$ are $False$?
  \item Why is $p \rightarrow q$ $True$ if $p$ is $False$ and $q$ is $True$?
\end{itemize}

If you have these questions, you are definitely not alone. Many people have the
same questions, too:
\begin{itemize}
  \item \href{https://philosophy.stackexchange.com/q/26719/44172}{Shouldn't statements be considered equivalent based on their meaning rather than truth tables?}
  \item \href{https://philosophy.stackexchange.com/q/34082/44172}{Why are conditionals with false antecedents considered true?}
  \item \href{https://math.stackexchange.com/q/3098664/665777}{How Implication or Material/Concrete Conditional works when the antecedent is false and the consequent is true}
  \item \href{https://math.stackexchange.com/q/70736/665777}{In classical logic, why is ($p \rightarrow q$) True if p is False and q is True?}
  \item etc.
\end{itemize}

On those Q \& A websites, some answers used ``Ex falso sequitur quodlibet''
(``from what is false any assertion validly follows''). This helps, to some
extent, but I think it still doesn't explain why it makes sense to define $p
\rightarrow q$ as $True$ in these two cases. In fact, I could argue the other way:
if $p$ is already $False$, then whatever it implies cannot make sense, even if
the proposition $q$ is a true statement, so the ``$\rightarrow$'' can't stand,
so it doesn't make sense to define it as $True$.

Among the answers I've read, \href{https://math.stackexchange.com/a/70739/665777}(this one)
is the closest to what I'm going to write below.

% =============================================================================
\section{A Quick Answer}
% =============================================================================

\colorbox{red}{TODO}

% =============================================================================
\section{What Does SEP Say?}
% =============================================================================

\textit{SEP-PL} establishes the basic framework for propositional logic and
then discuss the truth-functionality in this framework. The definition of
truth-functions naturally answers part of this question, and \textit{SEP-PL}
uses a separate subsection to discuss it further in order to provide additional
details.

% ******************************
\subsection{Basic Framework}
% ******************************

``Atomic'' propositions can be represented by \textbf{propositional variables}:
$p_1$, $p_2$, $p_3$, $\ldots$.

The \textbf{propositional connectives}, denoted by $c_1^1, c_2^1, \ldots, c_1^2,
c_2^2, \ldots$ can be applied to the propositions in order to create new
propositions.

In all these symbols, the subscripts are used to distinguish them from one
another. The superscripts indicate the connective's ``arity'', i.e., the number
of propositions they operate on.

The \textbf{formulas} of propositional logic are defined recursively by:
\begin{enumerate}
  \item Atomic propositional variables are \textbf{formulas}.
  \begin{itemize}
    \item \colorbox{lime}{Note}: \textit{SEP-PL} doesn't use this term but I prefer
    to call them \textbf{atomic formulas}.
  \end{itemize}

  \item If $c_n^m$ is a propositional connective, and $\langle A, B, C, \ldots
    \rangle$ is a sequence of $m$, possibly but not necessarily atomic (so each
    formula can be atomic or compound), possibly but not necessarily distinct
    (for example, $A$ can be equal to $B$), then the result of applying $c_n^m$
    to $\langle A, B, C, \ldots \rangle$ is a formula, written as \[c_n^m(A, B,
    C, \ldots)\].
    \begin{itemize}
      \item \colorbox{lime}{Note}: \textit{SEP-PL} doesn't use this term but I
        prefer to call them \textbf{compound formulas}.
      \item Examples:
      \begin{itemize}
        \item $p_4$
        \item $c_1^2(p_7, p_3)$
        \item $c_2^2(c_1^1(p_1), c_1^1(c_1^2(p_2, p_3)))$
      \end{itemize}
    \end{itemize}
\end{enumerate}

It is customary to the following:
\begin{itemize}
  \item Indicate the specific connectives with special characters, typically
    $\land$, $\lor$, $\supset$, $\lnot$.
  \item Use infix notation for binary connectives, such as $A \land B$ instead
    of $\land(A, B)$.
  \item Display parentheses only when there would be ambiguity. For example,
    if one wants to express the formula $(\lnot B) \land C$, one doesn't need
    the parentheses because the precedence guarantees the formula is calculated
    in the correct order. However, if one wants to express $\lnot (B \land C)$,
    the parentheses are necessary.
\end{itemize}

\colorbox{lime}{\textbf{Notes}}:
\begin{enumerate}
  \item This section is helpful for those who learn propositional logic for the
    first time. It defines the language of propositional logic in the most
    abstract way so the readers can see what propositional logic generally
    looks like. Understanding the abstraction is helpful when it gets more
    specific, because the abstract expressions provide the readers a full
    overview so the readers will not get lost in the details.
  \item Note that the possible values of a proposition is not defined in the
    framework. We usually think a proposition is either $True$ or $False$, but
    later when we examine the truth functions, you will see that having binary
    values is only one possible case (which is called the ``bivalent case'').
    Technically, a proposition may be in more than two states.
\end{enumerate}

% ******************************
\subsection{The interpretations}
% ******************************

``The logical analysis of propositional formulas proceeds from an association
of some meaning with the connectives.'' In other words, we want to understand
how we should interpret the meanings of the propositional connectives. As
\textit{SEP-PL} says:

\begin{displayquote}
  Because the original motivation for studying propositional logic is the
  observation that some particles in natural language often behave like
  propositional connectives, a natural first move is to try to specify formal
  rules of inference or precise assertability conditions that \textbf{capture}
  or \textbf{approximate} the role that these natural language predicates serve
  in informal reasoning.
\end{displayquote}

I find it interesting that \textit{SEP-PL} uses ``approximate''. It seems to
suggest that the logicians have already realized that people's thoughts are
by nature vague and difficult to be defined accurately on a mathematical level.
As a result, the mathematical tools may not always accurately reflect people's
thoughts. But it's more likely the other way: Mathematics is accurate; it is
people's thoughts and languages that are vague by nature, that they cannot
describe themselves in the way that can be fully and accurately expressed in
mathematics.

Nonetheless, the interpretations of the meanings of propositional connectives
can be divided into two major categories, with each category being further
divided into sub-categories:
\begin{itemize}
  \item The classical interpretation
    \begin{itemize}
      \item Truth functions
      \item Axiomatic deduction systems
    \end{itemize}
  \item The non-classical interpretation
    \begin{itemize}
      \item Multi-valued logics
      \item Constructive logics
      \item Relevance and connexive logics
      \item Linear logic
    \end{itemize}
\end{itemize}

In this article, I only talk about the truth functions in the classical
interpretation because, frankly speaking, I haven't learned the other
interpretations yet.

% ******************************
\subsection{Truth-functionality}
% ******************************

From an mathematical perspective, the truth functions are probably the simplest
form of interpretation to the propositional connectives.

Define $\mathcal{V} := \{v_i: v_i \in \mathcal{V}, i \in \mathbb{N}\}$ as a set
of truth values. A function \[f: \mathcal{V}^n \rightarrow \mathcal{V}\] that
maps a n-tuple of truth values to a truth value is called an \textbf{n-ary
truth function}.

\colorbox{lime}{Notes}: It's interesting to think about whether $\mathcal{V}$
can be an infinite set. When describing the basic framework, \textit{SEP-PL}
says: ``... the fact that we use natural numbers indicates the typical
convention that the vocabulary be countable.'' However, there is the concept of
``infinitely countable'' so a countable set can be infinite at the same time.
(I don't know the answer, but after quickly thinking about it, I think
$\mathcal{V}$ should be a finite set.)

Here are some additional definitions:
\begin{enumerate}
  \item \textbf{valence}: The number of values in $\mathcal{V}$ is called the
    valence of the function space.
  \item \textbf{bivalent}: When $\mathcal{V}$ has only two values, it is said
    to be ``bivalent''. When these two values happen to match the meanings of
    ``true'' or ``false'' in our daily conversation, denoted as \textbf{T} and
    \textbf{F} respectively, we have the classical theory of truth. (This is
    another case of going from ``abstract'' to ``concrete'': Most of the time,
    we talk about the ``true/false'' case, but that doesn't mean all the
    bivalent cases are all about ``true'' and ``false''.)
\end{enumerate}

The bivalent case is important because it is applicable in our daily life. So
we will examine this case further.

% ******************************
\subsection{Bivalent case: Boolean propositional logic}
% ******************************

% ------------------------------
\subsubsection{When n = 0, 1, 2}
% ------------------------------

\textit{SEP-PL} says:

\begin{displayquote}
  In the bivalent case, the truth values are denoted \textbf{T} and \textbf{F},
  and one has the classical theory of truth. Observe that in this classical
  function space are two \textbf{0-ary} truth functions which, following the
  conventions on indices for propositional connectives we could denote $f_1^0$
  and $f_2^0$, but are conventionally denoted $\top$ and $\bot$, defined by
  $\top = \textbf{T}$ and $\bot = \textbf{F}$.
\end{displayquote}

This is the case in which $n = 0$:

\begin{table}[H]
  \centering
  \begin{tabular}{|c|c|c|}
  \hline
  input   &   & meaning \\ [1ex] \hline
  $f_1^0$ & T & $\top$  \\ [0.5ex] \hline
  $f_2^0$ & F & $\bot$  \\ [0.5ex] \hline
  \end{tabular}
  \caption{0-ary truth functions}
\end{table}

When $n = 1$, the truth functions are \textbf{unary}:

\begin{table}[H]
  \centering
  \begin{tabular}{|c|c|c|c|c}
  \cline{1-4}
  input   & $\langle T \rangle$ & $\langle F \rangle$ & Meaning &  \\ [1ex] \cline{1-4}
  $f_1^1$ & T                   & T                   & $\top$  &  \\ [0.5ex] \cline{1-4}
  $f_2^1$ & T                   & F                   & Same    &  \\ [0.5ex] \cline{1-4}
  $f_3^1$ & F                   & T                   & NOT     &  \\ [0.5ex] \cline{1-4}
  $f_4^1$ & F                   & F                   & $\bot$  &  \\ [0.5ex] \cline{1-4}
  \end{tabular}
  \caption{unary truth functions}
\end{table}

When $n = 2$, the truth functions are \textbf{binary}:

\begin{table}[H]
  \centering
  \begin{tabular}{|c|c|c|c|c|c|}
  \hline
  input      & $\langle T, T \rangle$ & $\langle T, F \rangle$ & $\langle F, T \rangle$ & $\langle F, F \rangle$ & Meaning \\ [1ex] \hline
  $f_1^2$    & T                      & T                      & T                      & T                      &         \\ [0.5ex] \hline
  $f_2^2$    & T                      & T                      & T                      & F                      & OR      \\ [0.5ex] \hline
  $f_3^2$    & T                      & T                      & F                      & T                      &         \\ [0.5ex] \hline
  $f_4^2$    & T                      & T                      & F                      & F                      &         \\ [0.5ex] \hline
  $f_5^2$    & T                      & F                      & T                      & T                      &         \\ [0.5ex] \hline
  $f_6^2$    & T                      & F                      & T                      & F                      &         \\ [0.5ex] \hline
  $f_7^2$    & T                      & F                      & F                      & T                      & Same    \\ [0.5ex] \hline
  $f_8^2$    & T                      & F                      & F                      & F                      & AND     \\ [0.5ex] \hline
  $f_9^2$    & F                      & T                      & T                      & T                      &         \\ [0.5ex] \hline
  $f_{10}^2$ & F                      & T                      & T                      & F                      & XOR     \\ [0.5ex] \hline
  $f_{11}^2$ & F                      & T                      & F                      & T                      &         \\ [0.5ex] \hline
  $f_{12}^2$ & F                      & T                      & F                      & F                      &         \\ [0.5ex] \hline
  $f_{13}^2$ & F                      & F                      & T                      & T                      &         \\ [0.5ex] \hline
  $f_{14}^2$ & F                      & F                      & T                      & F                      &         \\ [0.5ex] \hline
  $f_{15}^2$ & F                      & F                      & F                      & T                      &         \\ [0.5ex] \hline
  $f_{16}^2$ & F                      & F                      & F                      & F                      &         \\ [0.5ex] \hline
  \end{tabular}
  \caption{binary truth functions}
\end{table}

% ------------------------------
\subsubsection{NOT, OR, AND, and their properties}
% ------------------------------

Let's only consider $f_3^1$ (logical NOT), $f_2^2$ (logical OR), $f_8^2$
(logical AND). In this way, the language of propositional logic restricted to
the connectives $\lnot$, $\land$, and $\lor$ with the formulas of two-element
Boolean algebra of \textbf{complementation}, \textbf{union}, and \textbf{
intersection}. Suppose we have two compound propositions $p$ and $q$, we can
define:
\begin{enumerate}
  \item $p$ is \textbf{classical propositional validity} if it evaluates as
    \textbf{T} on every possible assignment of values to its atomic
    propositional variables.
  \item $p$ is \textbf{classically satisfiable} if it evaluates as \textbf{T}
    on \textbf{at least one possible assignment} of values to its atomic
    propositional variables (and is \textbf{classically unsatisfiable}
    otherwise, i.e, none of the assignments can make $p$ be evaluated as
    \textbf{T}).
  \item $p$ is a \textbf{classical propositional consequence} of $q$ if on no
    assignment of values to the atoms that occur in $p$ and $q$ on which $q$,
    evaluates as \textbf{T} does $p$ evaluate as \textbf{F}.
    \colorbox{red}{TODO}: I don't understand this sentence.
  \item $p$ is a \textbf{classical propositional equivalent} of $q$ if $p$ and
    $q$ evaluate as \textbf{T} on precisely the \textbf{same assignments} of
    values to atoms.
\end{enumerate}

And the Boolean laws are applicable:
\begin{itemize}
  \item \textbf{interchange}: Suppose $p$, $q$, $r_i$, $m_1$, and $m_2$ are all
    formulas and:
    \begin{enumerate}
      \item $p$ and $q$ are equivalent.
      \item $m_1 = c_1^n(p, r_1, r_2, \ldots)$
      \item $m_2 = c_1^n(q, r_1, r_2, \ldots)$ (i.e., $m_2$ is only different
        from $m_1$ in the replacement of $p$ with $q$ and everything else is
        the same)
    \end{enumerate}
    Then $m_1$ and $m_2$ are also equivalent.
  \item \textbf{substitution}: Suppose $p$, $q$, $m_1$, $m_2$, $n_1$, and $n_2$
    are all formulas and:
    \begin{enumerate}
      \item $m_1 = c_1^n(p, \ldots)$
      \item $n_1 = c_2^n(p, \ldots)$
      \item $m_2$ is constructed by replacing $p$ in $m_1$ with $q$.
      \item $n_2$ is constructed by replacing $p$ in $n_1$ with $q$.
    \end{enumerate}
    Then:
    \begin{enumerate}
      \item $m_1$ is valid $\Rightarrow$ $m_2$ is valid.
      \item $m_1$ is unsatisfiable $\Rightarrow$ $m_2$ is unsatisfiable.
      \item $n_1$ is a consequence of $m_1$ $\Rightarrow$ $n_2$ is a consequence
        of $m_2$.
      \item $m_1 \equiv n_1$ $\Rightarrow$ $m_2 \equiv n_2$
    \end{enumerate}
  \item \textbf{complementation}: $p \lor \lnot p$ is classical validity (i.e.,
    always true). This is also called ``law of excluded middle'' (LEM).
  \item \textbf{double complementation}: $\lnot \lnot p \equiv p$
  \item \textbf{commutativity}
    \begin{enumerate}
      \item $p \land q \equiv q \land p$
      \item $p \lor q \equiv q \lor p$
    \end{enumerate}
  \item \textbf{associativity}
    \begin{enumerate}
      \item $(p \land q) \land r \equiv p \land (q \land r)$
      \item $(p \lor q) \lor r \equiv p \lor (q \lor r)$
    \end{enumerate}
  \item \textbf{distribution}
    \begin{enumerate}
      \item $p \land (q_1 \lor q_2 \lor \ldots \lor q_n) \equiv (p \land q_1)
        \lor (p \land q_2) \lor \ldots \lor (p \land q_n)$
      \item $p \lor (q_1 \land q_2 \land \ldots \land q_n) \equiv (p \lor q_1)
        \land (p \lor q_2) \land \ldots \land (p \lor q_n)$
    \end{enumerate}
  \item \textbf{De Morgan equivalence}
    \begin{enumerate}
      \item $\lnot (p_1 \land p_2 \land \ldots \land p_n) \equiv \lnot p_1
        \lor \lnot p_2 \lor \ldots \lor \lnot p_n$
      \item $\lnot (p_1 \lor p_2 \lor \ldots \lor p_n) \equiv \lnot p_1
        \land \lnot p_2 \land \ldots \land \lnot p_n$
    \end{enumerate}
\end{itemize}

% ------------------------------
\subsubsection{Normal forms}
% ------------------------------

\colorbox{red}{TODO}

% ------------------------------
\subsubsection{The material conditional}
% ------------------------------

\colorbox{red}{TODO}

\end{document}
