\documentclass[12pt, letterpaper, oneside]{book}
\usepackage{amsmath}
\usepackage{amssymb}
\usepackage{csquotes}
\usepackage{empheq}
\usepackage[letterpaper, textwidth=7.5in, textheight=8in]{geometry}
\usepackage{parskip}
\usepackage{xcolor}

\title{Exercises in \textit{Basic Analysis I: Introduction to Real Analysis, Volume 1}}
\author{Yaobin Wen}
\date{Jan 2024}

\begin{document}

\maketitle
\tableofcontents

% =============================================================================
%
% Chapter 1: Real Numbers
%
% =============================================================================

\chapter{Real Numbers}

% =============================================================================
\section{Exercises 1.1.1}
% =============================================================================

% ******************************
\subsection{Exercise 1.1.1}
% ******************************

\textbf{Proof}: This exercise requires to prove part (iii) of Proposition 1.1.8. That means I can make use of part (i)
and part (ii) in Proposition 1.1.8. The proof steps are as follows:
\begin{itemize}
  \item (a) $\because x < 0, \therefore (-x) > 0, \therefore (-x)y < (-x)z$ (Proposition 1.1.8 (i), (ii))
  \item (b) $\because (-x)y < (-x)z, \therefore (-x)y + xy < (-x)z + xy$ (Definition 1.1.7 (i))
  \item (c) $(-x)y + xy = (-x + x)y = 0y = 0$ (Definition 1.1.5 (D), (A5), (A2))
  \item (d) $\therefore 0 < (-x)z + xy$. ((b), (c))
  \item (e) $\therefore 0 + xz < (-x)z + xy + xz$ (Definition 1.1.7 (i))
  \item (f) $0 + xz = xz$ (Definition 1.1.5 (A4))
  \item (g) $(-x)z + xy + xz = \left((-x)z + xy\right) + xz = \left(xy + (-x)z\right) + xz =
          xy + \left((-x)z + xz \right) = xy + \left((-x) + x\right)z = xy + 0z = xy + 0 = xy$ (Definition 1.1.5 (D),
        (A2), (A3), (A4))
  \item (h) $\therefore xz < xy; \therefore xy > xz$ ((f), (g))
\end{itemize}

% ******************************
\subsection{Exercise 1.1.2}
% ******************************

% ------------------------------
\subsubsection{Description}
% ------------------------------

Let $S$ be an ordered set. Let $A \subset S$ be a nonempty finite subset. Then $A$ is bounded. Furthermore, $inf A$
exists and is in $A$ and $sup A$ exists and is in $A$. Hint: Use induction.

% ------------------------------
\subsubsection{Proof}
% ------------------------------

\textbf{Proof}: Let's use induction to prove it.

Because $A$ is a nonempty finite set, that means $A$ has at least one element.

\textbf{(1)}: Let's examine the case in which $A$ has exactly one element. Suppose $A = \{ a_1 \}$, then we know
$a_1 \in S$. Because $S$ is an ordered set, so $a_1 = a_1$. By the definition of $\le$, we can also say $a_1 \le a_1$.
Then by definition of \textit{bounded above}, we can see that there exists $a_1 \in S$ such that $\forall x in A$,
$x \le a_1$. Therefore, $A$ is bounded above and $a_1$ is an upper bound of $A$.

Similarly, by the definition of $\ge$, we can also say $a_1 \ge a_1$. Then by definition of \textit{bounded below}, we
can see that there exists $a_1 \in S$ such that $\forall x in A, x \ge a_1$. Therefore, $A$ is bounded below and $a_1$
is a lower bound of $A$.

In sum, $A$ is bounded when $A$ has exactly one element.

Now let's prove that $a_1$ is $A$'s supremum. Because $S$ is an ordered set, $\forall s \in S$, we know exactly one of
$s < a_1$, $s = a_1$, or $s > a_1$ holds. If $s < a_1$, by definition of \textit{upper bound}, we know $s$ is not an
upper bound of $A$. if $s > a_1$, we know $s$ is an upper bound of $A$. But because $s > a_1$, $s$ can't be equal to
$a_1$, so this $s$ must not be in $A$. When $s = a_1$, that means $s$ is $a_1$ itself, and by the definition of upper
bound, we can tell $a_1$ is also an upper bound. Therefore, $a_1$ is the upper bound of $A$ that whenever $s$ is any
upper bound for $A$, we have $a_1 \le s$. So we can see $a_1$ is the supremum of $A$. Because $a_1 \in A$, we can see
$sup A \in A$.

Similarly, we can prove that $inf A \in A$.

\textbf{(2)}: Let's assume that when $A$ has $N$ elements where $N \in \mathbb{N} \land N > 1$, $A$ is bounded and
$inf A \in A$ and $sup A \in A$. Let's define $A = \{ a_1, a_2, \ldots, a_N \}$. Without losing the generality, let's
order them in the ascending order so $a_1 \le a_2 \le \ldots \le a_N$. By the definition of upper bound and lower bound,
we can also tell that $inf A = a_1$ and $sup A = a_N$.

\textbf{(3)}: Now let's look at the case in which $A$ has $N+1$ elements where $N \in \mathbb{N} \land N > 1$. We can
construct this new $A$ by adding exactly one more element $a_{N+1}$ into the $A$ in step (2). Then:
\begin{itemize}
  \item If $a_{N+1} \le a_1$, then $inf A = a_{N+1}$ and $sup A = a_N$.
  \item If $a_1 \le a_{N+1} \le a_N$, then $inf A = a_1$ and $sup A = a_N$.
  \item If $a_N \le a_{N+1}$, then $inf A = a_1$ and $sup A = a_{N+1}$.
\end{itemize}

In all the cases above, $A$ is bounded and $inf A \in A$ and $sup A \in A$.

% ******************************
\subsection{Exercise 1.1.3}
% ******************************

\textbf{Proof}:
\begin{itemize}
  \item (a) $\because x > 0 \land x < y$, $\therefore xx < xy$ (Proposition 1.1.8 (ii))
  \item (b) $\because y > 0 \land x < y$, $\therefore yx < yy$ (Proposition 1.1.8 (ii))
  \item (c) $xy = yx$ (Definition 1.1.5 (M2))
  \item (d) $\because xx < xy \land xy = yx \land yx < yy$, $\therefore xx < yy$ (Definition 1.1.1 (ii))
  \item (e) Denote $xx$ as $x^2$ and $yy$ as $y^2$, we have $x^2 < y^2$.
\end{itemize}

% ******************************
\subsection{Exercise 1.1.4}
% ******************************

% ------------------------------
\subsubsection{Description}
% ------------------------------

Let $S$ be an ordered set. Let $B \subset S$ be bounded (above and below). Let $A \subset B$ be a nonempty subset.
Suppose all the \textit{inf}s and \textit{sup}s exist. Show that $inf B \le inf A \le sup A \le sup B$.

% ------------------------------
\subsubsection{Proof}
% ------------------------------

\textbf{(1)} First of all, let's prove $inf A \le sup A$:
\begin{itemize}
  \item (a) $A \ne \emptyset \rightarrow \exists a \in A$.
  \item (b) $IsUpperBound(sup A, A, S) \rightarrow (\forall a \in A. (a \le sup A))$ (Definition 1.1.2 (i))
  \item (c) $IsLowerBound(inf A, A, S) \rightarrow (\forall a \in A. (inf A \le a))$ (Definition 1.1.2 (ii))
  \item (d) $(b) \land (c) \rightarrow inf A \le a \le sup A$
\end{itemize}

\textbf{(2)} Secondly, let's prove $inf B \le inf A$:
\begin{itemize}
  \item (a) $IsLowerBound(inf B, B, S) \rightarrow (\forall b \in B. (inf B \le b))$ (Definition 1.1.2 (ii))
  \item (b) $IsLowerBound(inf A, A, S) \rightarrow (\forall a \in A. (inf A \le a))$ (Definition 1.1.2 (ii))
  \item (c) $A \subset B \rightarrow \forall a \in A. (a \in B)$
  \item (d) $(a) \land (c) \rightarrow \forall a \in A. (inf B \le a)$
  \item (e) $(A \subset S) \land (inf B \in S) \land (d) \rightarrow IsLowerBound(inf B, A, S)$ (Definition 1.1.2 (ii))
  \item (f) $IsLowerBound(inf B, A, S) \land inf A \rightarrow inf B \le inf A$ (Definition 1.1.2 (iv))
\end{itemize}

\textbf{(3)} Lastly, let's prove $sup A \le sup B$:
\begin{itemize}
  \item (a) $IsUpperBound(sup B, B, S) \rightarrow (\forall b \in B. (b \le sup B))$ (Definition 1.1.2 (i))
  \item (b) $IsUpperBound(sup A, A, S) \rightarrow (\forall a \in A. (a \le sup A))$ (Definition 1.1.2 (i))
  \item (c) $A \subset B \rightarrow \forall a \in A. (a \in B)$
  \item (d) $(a) \land (c) \rightarrow \forall a \in A. (a \le sup B)$
  \item (e) $(A \subset S) \land (sup B \in S) \land (d) \rightarrow IsUpperBound(sup B, A, S)$ (Definition 1.1.2 (i))
  \item (f) $IsUpperBound(sup B, A, S) \land sup A \rightarrow sup A \le sup B$
\end{itemize}

In sum, $inf B \le inf A \le sup A \le sup B$.

% ******************************
\subsection{Exercise 1.1.5}
% ******************************

% ------------------------------
\subsubsection{Description}
% ------------------------------

Let $S$ be an ordered set. Let $A \subset S$ and suppose $b$ is an upper bound for $A$. Suppose $b \in A$. Show that
$b = sup A$.

% ------------------------------
\subsubsection{Proof}
% ------------------------------

\begin{itemize}
  \item (a) $sup A \rightarrow \forall x \in A. (x \le sup A)$ (Definition 1.1.2 (i))
  \item (b) $(b \in A) \land (a) \rightarrow b \le sup A$
  \item (c) $IsUpperBound(b, A, S) \land sup A \rightarrow sup A \le b$ (Definition 1.1.2 (iii))
  \item (d) $(b) \land (c) \rightarrow b = sup A$ (Definition 1.1.1 (i))
\end{itemize}

% ******************************
\subsection{Exercise 1.1.6}
% ******************************

% ------------------------------
\subsubsection{Description}
% ------------------------------

Let $S$ be an ordered set. Let $A \subset S$ be a nonempty subset that is bounded above. Suppose $sup A$ exists and
$sup A \notin A$. Show that $A$ contains a countably infinite subset.

% ------------------------------
\subsubsection{Proof}
% ------------------------------

First of all, we must understand what this exercise is asking us to prove. It is asking us to prove that:
\begin{enumerate}
  \item $A$ contains a subset.
  \item And, this subset is infinite.
  \item And, this subset is countable.
\end{enumerate}

This exercise is \textbf{NOT} asking us to prove that:
\begin{itemize}
  \item $A$ is countably infinite.
  \item Or, $A$ has infinite subsets.
\end{itemize}

\textbf{(1)}: Let's prove $A$ is infinite:
\begin{itemize}
  \item (a) For the sake of contradiction, assume $A$ is finite.
  \item (b) According to Exercise 1.1.2, $sup A \in A$.
  \item (c) But we already know $sup A \notin A$.
  \item (d) (b) and (c) contradict, so the assumption (a) must be invalid. So $A$ is not finite, hence infinite.
\end{itemize}

\textbf{(2)}: Let's prove that $A$ contains an infinite subset:
\begin{itemize}
  \item (a) $A$ is infinite. (Part (1))
  \item (b) $A \subset A$.
  \item (c) $A$ contains at least one subset, $A$ itself, that's infinite. Let's call this subset $B$.
\end{itemize}

\textbf{(3)}: \colorbox{red}{\textcolor{yellow}{TODO(ywen):}} Let's prove that $B$ is countable.

% ******************************
\subsection{Exercise 1.1.7}
% ******************************

% ------------------------------
\subsubsection{Description}
% ------------------------------

Find a (nonstandard) ordering of the set of natural numbers $\mathbb{N}$ such that there exists a nonempty proper
subset $A \subsetneq N$ and such that $sup A \in \mathbb{N}$, but $sup A \notin A$. To keep things straight it might be
a good idea to use a different notation for the nonstandard ordering such as $n \prec m$.

% ------------------------------
\subsubsection{Remarks about the wrong analysis and solution}
% ------------------------------

The following two sections are a wrong analysis and a wrong solution when I was working on this exercise for the first
time. I wanted to still keep them here because I may learn from the wrong thoughts.

% ------------------------------
\subsubsection{Wrong analysis}
% ------------------------------

Note that in this textbook, the set of natural numbers $\mathbf{N}$ does not include zero.

Note that in the definition of upper bound (Definition 1.1.2 (i)), we need to compare every element in the subset $E$
with the possible upper bound value $b$ using the operator $\le$ (i.e., $< \lor =$).

Also note that in the standard ordering of the numbers in $\mathbb{N}$, $a = b$ if $a$ and $b$ are the same number. For
example, $1 = 1$, $79 = 79$.

Because the relation $=$ exists in the standard ordering, when we examine any upper bounded subset of $\mathbb{N}$, the
largest number in the subset, denoted $n_0$, always satisfies $n_0 \le n_0$ because $n_0 = n_0$. As a result, $n_0$ is
the supremum of this subset but $n_0$ is a member of this subset.

Therefore, the key is to remove the $=$ relation.

% ------------------------------
\subsubsection{Wrong solution}
% ------------------------------

With the help of the standard ordering relation $<$ in $\mathbb{N}$, let's define the nonstandard ordering relation
$\prec$ in $\mathbb{N}$ as follows. $\forall x, y \in \mathbb{N}$:
\begin{enumerate}
  \item $(x = y) \lor (x < y) \rightarrow x \prec y$
  \item $x > y \rightarrow x \succ y$
\end{enumerate}

We can prove that $\mathbb{N}$ is still an ordered set under the relation $\prec$:
\begin{itemize}
  \item $\forall x, y \in \mathbf{N}. (ExactlyOneIsTrue(x < y, x = y, x > y) \rightarrow ExactlyOneIsTrue(x \prec y, x \succ y))$
  \item $x \le y \rightarrow x \prec y$; $y \le z \rightarrow y \prec z$; $(x \le y) \land (y \le z) \rightarrow x \le z \rightarrow x \prec z$
\end{itemize}

\colorbox{red}{TODO(ywen)}: Interestingly, should $1 \prec 1$ and $1 \succ 1$ be treated as the same predicate or
different predicates? If they should be treated differently, then $\mathbb{N}$ and $\prec$ can't be an ordered set.

Then every proper subset $A = \{ k \in \mathbb{N}: 0, 1, 2, \ldots, k \}$ has $sup A = k + 1 \in \mathbb{N}$, but
$sup A = k+1 \notin A$. \colorbox{lime}{NOTE(ywen)}: I realized the solution was wrong when I arrived here. Clearly,
$sup A$ should be $k$ instead of $k+1$ because:
\begin{itemize}
  \item (a) $(\forall x \in A. (x \le k \land x \prec k)) \rightarrow IsUpperBound(k, A, \mathbb{N})$
  \item (b) $\forall y \in \mathbb{N}. (k \le y \rightarrow (\forall x \in A. (x \le y \land x \prec y)))$
  \item (c) $(b) \rightarrow (\forall y \in \mathbb{N}. (k \le y \rightarrow IsUpperBound(y, A, \mathbb{N})))$
  \item (d) $\forall y \in \mathbb{N}. (k \le y \rightarrow k \prec y)$
  \item $(a) \land (c) \land (d) \rightarrow sup A = k$
\end{itemize}

But $k \in A$, so $A$ is not the subset that satisfies the requirements of this exercise.

% ------------------------------
\subsubsection{Correct analysis}
% ------------------------------

Think about Example 1.1.4: The subset \{$x \in \mathbb{Q}: x^2 < 2$\} does not have a supremum in $\mathbb{Q}$. In this
example, we notice that this subset is bounded and has infinite elements. The rational numbers in this subset can get
\textbf{infinitely closer} to $\pm \sqrt{2}$ but can never reach them. The property of \textbf{``getting infinitely
  closer''} can give us the hint of finding the appropriate binary relation $\prec$.

% ------------------------------
\subsubsection{Correct solution}
% ------------------------------

With the help of the standard ordering relation $<$ in $\mathbb{N}$, let's define the nonstandard ordering relation
$\prec$ in $\mathbb{N}$ as follows. $\forall x, y \in \mathbb{N}$:
\begin{enumerate}
  \item $IsOdd(x) \land IsEven(y) \rightarrow x \prec y$
  \item $IsEven(x) \land IsOdd(y) \rightarrow x \succ y$
  \item $IsOdd(x) \land IsOdd(y) \rightarrow (x < y \rightarrow x \prec y) \lor (x = y \rightarrow x = y) \lor
          (x > y \rightarrow x \succ y)$
  \item $IsEven(x) \land IsEven(y) \rightarrow (x < y \rightarrow x \prec y) \lor (x = y \rightarrow x = y) \lor
          (x > y \rightarrow x \succ y)$
\end{enumerate}

In sum:
\begin{enumerate}
  \item All the odd numbers are less than the even numbers, e.g., $19337 \prec 2$.
  \item Within all the odd (or even) numbers, the numbers are ordered in the same way as the standard ordering, e.g.,
        $1 \prec 3 \prec 5 \prec 7 \ldots$, $2 \prec 4 \prec 6 \prec 8 \ldots$
\end{enumerate}

Then we examine the proper subset $A = \{ x \in \mathbb{N}: IsOdd(x) \}$:
\begin{itemize}
  \item (a): $IsEven(2) \land 2 \notin A$
  \item (b): $\forall e \in \mathbb{N}. (IsEven(e) \rightarrow (\forall a \in A. (a \prec e)))$
  \item (c): $(b) \rightarrow (\forall e \in \mathbb{N}. (IsEven(e) \rightarrow IsUpperBound(e, A, \mathbb{N})))$
  \item (d): $(a) \land (c) \rightarrow IsUpperBound(2, A, \mathbb{N})$
  \item (e): $\forall e \in \mathbb{N}. (IsEven(e) \rightarrow 2 \le e)$
  \item (f): $(c) \land (d) \land (e) \rightarrow sup A = 2$
\end{itemize}

Therefore, $A$ under this $\prec$ relation has $sup A$ but $sup A \notin A$.

% =============================================================================
%
% First-order logic predicates
%
% =============================================================================

\chapter{First-order logic predicates}

Throughout all the exercises, we use the following first-order logic predicates in the proofs:

\begin{itemize}
  \item $ExactlyOneIsTrue(pred1, pred2, \ldots, predN)$: Exactly one of the predicates of \{ pred1, pred2, $\ldots$,
        predN \} is true.
  \item $IsBijection(f, A, B)$: Function $f$ is a bijection from the set $A$ to set $B$.
  \item $IsOdd(i)$: Integer $i$ is an odd number (e.g., $-79$, $3$).
  \item $IsEven(i)$: Integer $i$ is an even number (e.g., $-880$, $0$, $1938$).
  \item $IsLowerBound(e, E, S)$: Element $e$ is an lower bound of $E$ in $S$ where $E \subset S$.
  \item $IsUpperBound(e, E, S)$: Element $e$ is an upper bound of $E$ in $S$ where $E \subset S$.
\end{itemize}

\end{document}
