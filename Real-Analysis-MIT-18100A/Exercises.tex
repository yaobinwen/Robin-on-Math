\documentclass[12pt, letterpaper, oneside]{book}
\usepackage{amsfonts}
\usepackage{amsmath}
\usepackage{amssymb}
\usepackage{csquotes}
\usepackage{empheq}
\usepackage{float}
\usepackage{hyperref}
\usepackage[letterpaper, textwidth=7.5in, textheight=8in]{geometry}
\hypersetup{
  colorlinks=true,
  linkcolor=blue,
  filecolor=magenta,
  urlcolor=blue,
}
\usepackage{parskip}
\usepackage{pseudocode}
\usepackage{xcolor}

\title{Exercises in \textit{Basic Analysis I: Introduction to Real Analysis, Volume 1}}
\author{Yaobin Wen}
\date{Jan 2024}

\begin{document}

\maketitle
\tableofcontents

% =============================================================================
%
% Chapter 1: Real Numbers
%
% =============================================================================

\chapter{Real Numbers}

% =============================================================================
\section{Exercises 1.1.1}
% =============================================================================

% ******************************
\subsection{Exercise 1.1.1}
% ******************************

% ------------------------------
\subsubsection{Description}
% ------------------------------

Prove part (iii) of Proposition 1.1.8. That is, let $F$ be an ordered field and $x, y, z \in F$. Prove: If $x < 0$ and
$y < z$, then $xy > xz$.

% ------------------------------
\subsubsection{Proof}
% ------------------------------

This exercise requires to prove part (iii) of Proposition 1.1.8. That means I can make use of part (i) and part (ii) in
Proposition 1.1.8. The proof steps are as follows:
\begin{itemize}
  \item (a) $\because x < 0, \therefore (-x) > 0, \therefore (-x)y < (-x)z$ (Proposition 1.1.8 (i), (ii))
  \item (b) $\because (-x)y < (-x)z, \therefore (-x)y + xy < (-x)z + xy$ (Definition 1.1.7 (i))
  \item (c) $(-x)y + xy = (-x + x)y = 0y = 0$ (Definition 1.1.5 (D), (A5), (A2))
  \item (d) $\therefore 0 < (-x)z + xy$. ((b), (c))
  \item (e) $\therefore 0 + xz < (-x)z + xy + xz$ (Definition 1.1.7 (i))
  \item (f) $0 + xz = xz$ (Definition 1.1.5 (A4))
  \item (g) $(-x)z + xy + xz = \left((-x)z + xy\right) + xz = \left(xy + (-x)z\right) + xz =
          xy + \left((-x)z + xz \right) = xy + \left((-x) + x\right)z = xy + 0z = xy + 0 = xy$ (Definition 1.1.5 (D),
        (A2), (A3), (A4))
  \item (h) $\therefore xz < xy; \therefore xy > xz$ ((f), (g))
\end{itemize}

$\blacksquare$

% ******************************
\subsection{Exercise 1.1.2}
% ******************************

% ------------------------------
\subsubsection{Description}
% ------------------------------

Let $S$ be an ordered set. Let $A \subset S$ be a nonempty finite subset. Then $A$ is bounded. Furthermore, $inf A$
exists and is in $A$ and $sup A$ exists and is in $A$. Hint: Use induction.

% ------------------------------
\subsubsection{Breakdown}
% ------------------------------

This exercise consists of many smaller parts, so let me list all of them. Let $S$ be an ordered set. Let $A \subset S$
be a nonempty finite subset. Then:
\begin{enumerate}
  \item[(1a)] $A$ is bounded above.
  \item[(1b)] $sup A$ exists.
  \item[(1c)] $sup A \in A$.
  \item[(2a)] $A$ is bounded below.
  \item[(2b)] $inf A$ exists.
  \item[(2c)] $inf A \in A$.
\end{enumerate}

I will prove with non-induction and then with induction.

% ------------------------------
\subsubsection{Lemma 1.1.2 (A): A nonempty, (totally) ordered, finite set has the minimum and the maximum}
% ------------------------------

Consider the set $S$:
\begin{enumerate}
  \item $S \ne \emptyset$.
  \item $IsChain(S)$ (i.e., totally ordered).
  \item $IsFinite(S)$ (i.e., the algorithm can terminate).
\end{enumerate}

The following algorithm can find its minimum and its maximum:

\begin{pseudocode}[ruled]{FindMinimumMaximum}{S}
  \LOCAL{min, max} \\
  e \GETS \mbox{pick a random element from S} \\
  min \GETS e \\
  max \GETS e \\
  S \GETS S \setminus \{e\} \\
  \\
  \WHILE (S \ne \emptyset) \DO
  \BEGIN
  e \GETS \mbox{pick a random element from S} \\
  \IF e < min \THEN min \GETS e \\
  \IF e > max \THEN max \GETS e \\
  S \GETS S \setminus \{e\} \\
  \END \\
  \\
  \RETURN{min, max}
\end{pseudocode}

We can tell that:
\begin{enumerate}
  \item $min \in S$
  \item $max \in S$
  \item $\forall e \in S. (min \leqslant e \leqslant max)$.
\end{enumerate}

% ------------------------------
\subsubsection{Proof (Non-induction)}
% ------------------------------

Definition 1.1.2 defines ``upper bound'' and ``lower bound'' on a subset of the (totally) ordered
set $S$. In Lemma 1.1.2 (A) above, we only thought about a totally ordered set $S$. However, because
any set is a subset of itself, we can consider $S$ itself as a subset of $S$. Therefore, given
$IsChain(S) \land S \subset S$:
\begin{enumerate}
  \item[(1)] $min, max \leftarrow FindMinimumMaximum(S)$ (Lemma 1.1.2 (A))
  \item[(2)] $(max \in S \land (\forall x \in S. (x \leqslant max))) \rightarrow IsUpperBound(max, S, S)$.
  \item[(3)] Suppose $\exists b \in S. (IsUpperBound(b, S, S) \land b < max)$:
        \begin{enumerate}
          \item $\because IsUpperBound(b, S, S) \therefore \forall x \in S. (x \leqslant b)$
          \item $\because max \in S, \therefore max \leqslant b$
          \item $\therefore (b < max \land max \leqslant b)$. Contradiction.
          \item $\therefore \lnot \exists b \in S. (IsUpperBound(b, S, S) \land b < max)$
          \item $\therefore \forall b. (IsUpperBound(b, S, S) \rightarrow max \leqslant b)$
          \item $\therefore sup S = max$
        \end{enumerate}
  \item[(4)] $(min \in S \land (\forall x \in S. (x \geqslant min))) \rightarrow IsLowerBound(min, S, S)$.
  \item[(5)] Suppose $\exists b \in S. (IsLowerBound(b, S, S) \land b > min)$:
        \begin{enumerate}
          \item $\because IsLowerBound(b, S, S) \therefore \forall x in S. (x \geqslant b)$
          \item $\because min \in S \therefore min \geqslant b$
          \item $\therefore (b > min \land min \geqslant b)$. Contradiction.
          \item $\therefore \lnot \exists b \in S. (IsLowerBound(b, S, S) \land b > min)$
          \item $\therefore \forall b. (IsLowerBound(b, S, S) \rightarrow min \geqslant b)$
          \item $\therefore inf S = min$
        \end{enumerate}
\end{enumerate}

% ------------------------------
\subsubsection{Proof (Induction)}
% ------------------------------

The exercise suggests induction. To use induction, we can describe the original problem in the form of ``P(n)'' where
$n \in \mathbb{N}$ as follows:

\begin{displayquote}
  Let $S$ be an ordered set. Let $A = \{ a_n \in S: n \in \mathbb{N} \land n \ge 1 \}$. Then ...
\end{displayquote}

Now let's use induction to prove them one by one.

\textbf{P(1) basis}:
\begin{enumerate}
  \item[(1)] $A = \{a_1\}$
  \item[(2)] $\exists a_1 \in A. (\forall x in A. (x \leqslant a_1))$.
  \item[(3)] $(2) \rightarrow (IsBoundedAbove(A, A) \land IsUpperBound(a_1, A, A) \land sup A = a_1)$.
  \item[(4)] $\exists a_1 \in A. (\forall x in A. (x \geqslant a_1))$.
  \item[(5)] $(4) \rightarrow (IsBoundedBelow(A, A) \land IsLowerBound(a_1, A, A) \land inf A = a_1)$.
\end{enumerate}

\textbf{P(1+1) basic of extension}:
\begin{enumerate}
  \item[(1)] Add a new element $a_2$ to $A$ in P(1).
  \item[(2)] If $a_1 < a_2$, then it's easy to tell that $A'=\{a_1, a_2\}$ is bounded above and below, and $inf A' = a_1$
        and $sup A' = a_2$.
  \item[(3)] If $a_2 < a_1$, then define $b_1 = a_2$, $b_2 = a_1$, and let $A'=\{b_1, b_2\}$. It's easy to tell that $A'$
        is bounded above and below, and $inf A' = b_1 = a_2$ and $sup A' = b_2 = a_1$.
\end{enumerate}

\textbf{P(1+k) extension}: In fact, if we repeat the process above $(k-1)$ times, we can construct P(1+k), i.e.,
$A = \{a_1, a_2, \ldots, a_{1+k}\}$ where $inf A = a_1$ and $sup A = a_{1+k}$.

\textbf{P(1+k+1)}:If we repeat the same process one more time over P(1+k), we can still construct P(1+k+1), i.e.,
$A = \{a_1, a_2, \ldots, a_{1+k}, a_{1+k+1}\}$ where $inf A = a_1$ and $sup A = a_{1+k+1}$.

$\blacksquare$

% ******************************
\subsection{Exercise 1.1.3}
% ******************************

% ------------------------------
\subsubsection{Description}
% ------------------------------

Prove part (vi) of Proposition 1.1.8. That is, let $x, y \in F$, where $F$ is an ordered field, such that $0 < x < y$.
Show that $x^2 < y^2$.

% ------------------------------
\subsubsection{Proof}
% ------------------------------

\begin{enumerate}
  \item[(a)] $\because x > 0 \land x < y$, $\therefore xx < xy$ (Proposition 1.1.8 (ii))
  \item[(b)] $\because y > 0 \land x < y$, $\therefore yx < yy$ (Proposition 1.1.8 (ii))
  \item[(c)] $xy = yx$ (Definition 1.1.5 (M2))
  \item[(d)] $\because xx < xy \land xy = yx \land yx < yy$, $\therefore xx < yy$ (Definition 1.1.1 (ii))
  \item[(e)] Denote $xx$ as $x^2$ and $yy$ as $y^2$, we have $x^2 < y^2$.
\end{enumerate}

$\blacksquare$

% ******************************
\subsection{Exercise 1.1.4}
% ******************************

% ------------------------------
\subsubsection{Description}
% ------------------------------

Let $S$ be an ordered set. Let $B \subset S$ be bounded (above and below). Let $A \subset B$ be a nonempty subset.
Suppose all the \textit{inf}s and \textit{sup}s exist. Show that $inf B \le inf A \le sup A \le sup B$.

% ------------------------------
\subsubsection{Proof}
% ------------------------------

\textbf{(1)} First of all, let's prove $inf A \le sup A$:
\begin{itemize}
  \item (a) $A \ne \emptyset \rightarrow \exists a \in A$.
  \item (b) $IsUpperBound(sup A, A, S) \rightarrow (\forall a \in A. (a \le sup A))$ (Definition 1.1.2 (i))
  \item (c) $IsLowerBound(inf A, A, S) \rightarrow (\forall a \in A. (inf A \le a))$ (Definition 1.1.2 (ii))
  \item (d) $(b) \land (c) \rightarrow inf A \le a \le sup A$
\end{itemize}

\textbf{(2)} Secondly, let's prove $inf B \le inf A$:
\begin{itemize}
  \item (a) $IsLowerBound(inf B, B, S) \rightarrow (\forall b \in B. (inf B \le b))$ (Definition 1.1.2 (ii))
  \item (b) $IsLowerBound(inf A, A, S) \rightarrow (\forall a \in A. (inf A \le a))$ (Definition 1.1.2 (ii))
  \item (c) $A \subset B \rightarrow \forall a \in A. (a \in B)$
  \item (d) $(a) \land (c) \rightarrow \forall a \in A. (inf B \le a)$
  \item (e) $(A \subset S) \land (inf B \in S) \land (d) \rightarrow IsLowerBound(inf B, A, S)$ (Definition 1.1.2 (ii))
  \item (f) $IsLowerBound(inf B, A, S) \land inf A \rightarrow inf B \le inf A$ (Definition 1.1.2 (iv))
\end{itemize}

\textbf{(3)} Lastly, let's prove $sup A \le sup B$:
\begin{itemize}
  \item (a) $IsUpperBound(sup B, B, S) \rightarrow (\forall b \in B. (b \le sup B))$ (Definition 1.1.2 (i))
  \item (b) $IsUpperBound(sup A, A, S) \rightarrow (\forall a \in A. (a \le sup A))$ (Definition 1.1.2 (i))
  \item (c) $A \subset B \rightarrow \forall a \in A. (a \in B)$
  \item (d) $(a) \land (c) \rightarrow \forall a \in A. (a \le sup B)$
  \item (e) $(A \subset S) \land (sup B \in S) \land (d) \rightarrow IsUpperBound(sup B, A, S)$ (Definition 1.1.2 (i))
  \item (f) $IsUpperBound(sup B, A, S) \land sup A \rightarrow sup A \le sup B$
\end{itemize}

In sum, $inf B \le inf A \le sup A \le sup B$.

% ******************************
\subsection{Exercise 1.1.5}
% ******************************

% ------------------------------
\subsubsection{Description}
% ------------------------------

Let $S$ be an ordered set. Let $A \subset S$ and suppose $b$ is an upper bound for $A$. Suppose $b \in A$. Show that
$b = sup A$.

% ------------------------------
\subsubsection{Proof}
% ------------------------------

\begin{itemize}
  \item (a) $sup A \rightarrow \forall x \in A. (x \le sup A)$ (Definition 1.1.2 (i))
  \item (b) $(b \in A) \land (a) \rightarrow b \le sup A$
  \item (c) $IsUpperBound(b, A, S) \land sup A \rightarrow sup A \le b$ (Definition 1.1.2 (iii))
  \item (d) $(b) \land (c) \rightarrow b = sup A$ (Definition 1.1.1 (i))
\end{itemize}

% ******************************
\subsection{Exercise 1.1.6}
% ******************************

% ------------------------------
\subsubsection{Description}
% ------------------------------

Let $S$ be an ordered set. Let $A \subset S$ be a nonempty subset that is bounded above. Suppose $sup A$ exists and
$sup A \notin A$. Show that $A$ contains a countably infinite subset.

% ------------------------------
\subsubsection{Proof}
% ------------------------------

First of all, we must understand what this exercise is asking us to prove. It is asking us to prove that:
\begin{enumerate}
  \item $A$ contains a subset.
  \item And, this subset is infinite.
  \item And, this subset is countable.
\end{enumerate}

This exercise is \textbf{NOT} asking us to prove that:
\begin{itemize}
  \item $A$ is countably infinite.
  \item Or, $A$ has infinite subsets.
\end{itemize}

\textbf{Part (1)}: Let's prove $A$ is infinite:
\begin{enumerate}
  \item[(a)] For the sake of contradiction, assume $A$ is finite.
  \item[(b)] According to Exercise 1.1.2, $sup A \in A$.
  \item[(c)] But we already know $sup A \notin A$.
  \item[(d)] (b) and (c) contradict, so the assumption (a) must be invalid. So $A$ is not finite, hence infinite.
\end{enumerate}

\textbf{Part (2)}: Let's prove that $A$ contains an infinite subset. Note that this question asks us to prove this:
\[\exists B \subset A. (IsInfinite(B) \land IsCountable(B))\]
It does \textbf{NOT} ask us to prove $\forall B \subset A. (IsInfinite(B) \rightarrow IsCountable(B))$. Therefore, as
long as we can find/construct one countably infinite subset, we are done. To find this countably infinite subset, we
need to prove the following lemma:

\textbf{Lemma 1.1.6 (A)}: $\forall x \in A, \exists y \in A. (x \ne y \land x < y < sup A)$.

In other words, the lemma means: Given any $x$ in $A$, we can always find another element in $A$ that sits between $x$
and $supA$ and is different from both of them.

For the sake of contradiction, assume $\exists x_0 \in A, \lnot \exists y \in A. (x_0 \ne y \land x_0 < y < sup A)$. In
other words, $\forall y \in A. (y \leqslant x_0)$. According to Definition 1.1.2, $x_0$ is an upper bound of $A$. This
contradicts with $sup A$ in two ways:
\begin{enumerate}
  \item[(1)] $(IsUpperBound(x_0, A, S) \land x_0 < sup A)$ $\rightarrow$ $x_0$ should be the supremum, not the current
        $sup A$.
  \item[(2)] If $x_0$ is the supremum, that means $sup A \in A$, contradicting with the condition $sup A \notin A$.
\end{enumerate}

Therefore, Lemma 1.1.6 (A) holds.

With Lemma 1.1.6 (A) proven, we can construct a subset of $A$ as follows:

\begin{pseudocode}[ruled]{ConstructCountablyInfiniteSubset}{A}
  \LOCAL{i \GETS 1} \\
  T \GETS \emptyset \\
  t_i \GETS PickElement(A) \\
  T \GETS T \cup \{t_i\} \\
  \WHILE \TRUE \DO
  \BEGIN
  i \GETS i + 1 \\
  t_i \GETS PickElement(A, (e > t_{i-1})) \\
  T \GETS T \cup \{t_i\} \\
  \END
  \\
  \RETURN{T}
\end{pseudocode}

The algorithm above doesn't terminate, but it does construct an infinite set $T \subset A$, and every element in $T$ is
associated with a natural number in $\mathbb{N}$. Therefore, we can see $|T| = |\mathbb{N}|$. Therefore, $T \subset A$
is countably infinite.

% ******************************
\subsection{Exercise 1.1.7}
% ******************************

% ------------------------------
\subsubsection{Description}
% ------------------------------

Find a (nonstandard) ordering of the set of natural numbers $\mathbb{N}$ such that there exists a nonempty proper
subset $A \subsetneq N$ and such that $sup A \in \mathbb{N}$, but $sup A \notin A$. To keep things straight it might be
a good idea to use a different notation for the nonstandard ordering such as $n \prec m$.

% ------------------------------
\subsubsection{Remarks about the wrong analysis and solution}
% ------------------------------

The following two sections are a wrong analysis and a wrong solution when I was working on this exercise for the first
time. I wanted to still keep them here because I may learn from the wrong thoughts.

% ------------------------------
\subsubsection{Wrong analysis}
% ------------------------------

Note that in this textbook, the set of natural numbers $\mathbf{N}$ does not include zero.

Note that in the definition of upper bound (Definition 1.1.2 (i)), we need to compare every element in the subset $E$
with the possible upper bound value $b$ using the operator $\le$ (i.e., $< \lor =$).

Also note that in the standard ordering of the numbers in $\mathbb{N}$, $a = b$ if $a$ and $b$ are the same number. For
example, $1 = 1$, $79 = 79$.

Because the relation $=$ exists in the standard ordering, when we examine any upper bounded subset of $\mathbb{N}$, the
largest number in the subset, denoted $n_0$, always satisfies $n_0 \le n_0$ because $n_0 = n_0$. As a result, $n_0$ is
the supremum of this subset but $n_0$ is a member of this subset.

Therefore, the key is to remove the $=$ relation.

% ------------------------------
\subsubsection{Wrong solution}
% ------------------------------

With the help of the standard ordering relation $<$ in $\mathbb{N}$, let's define the nonstandard ordering relation
$\prec$ in $\mathbb{N}$ as follows. $\forall x, y \in \mathbb{N}$:
\begin{enumerate}
  \item $(x = y) \lor (x < y) \rightarrow x \prec y$
  \item $x > y \rightarrow x \succ y$
\end{enumerate}

We can prove that $\mathbb{N}$ is still an ordered set under the relation $\prec$:
\begin{itemize}
  \item $\forall x, y \in \mathbf{N}. (ExactlyOneIsTrue(x < y, x = y, x > y) \rightarrow ExactlyOneIsTrue(x \prec y, x \succ y))$
  \item $x \le y \rightarrow x \prec y$; $y \le z \rightarrow y \prec z$; $(x \le y) \land (y \le z) \rightarrow x \le z \rightarrow x \prec z$
\end{itemize}

\colorbox{red}{TODO(ywen)}: Interestingly, should $1 \prec 1$ and $1 \succ 1$ be treated as the same predicate or
different predicates? If they should be treated differently, then $\mathbb{N}$ and $\prec$ can't be an ordered set.

Then every proper subset $A = \{ k \in \mathbb{N}: 0, 1, 2, \ldots, k \}$ has $sup A = k + 1 \in \mathbb{N}$, but
$sup A = k+1 \notin A$. \colorbox{lime}{NOTE(ywen)}: I realized the solution was wrong when I arrived here. Clearly,
$sup A$ should be $k$ instead of $k+1$ because:
\begin{itemize}
  \item (a) $(\forall x \in A. (x \le k \land x \prec k)) \rightarrow IsUpperBound(k, A, \mathbb{N})$
  \item (b) $\forall y \in \mathbb{N}. (k \le y \rightarrow (\forall x \in A. (x \le y \land x \prec y)))$
  \item (c) $(b) \rightarrow (\forall y \in \mathbb{N}. (k \le y \rightarrow IsUpperBound(y, A, \mathbb{N})))$
  \item (d) $\forall y \in \mathbb{N}. (k \le y \rightarrow k \prec y)$
  \item $(a) \land (c) \land (d) \rightarrow sup A = k$
\end{itemize}

But $k \in A$, so $A$ is not the subset that satisfies the requirements of this exercise.

% ------------------------------
\subsubsection{Correct analysis}
% ------------------------------

Think about Example 1.1.4: The subset \{$x \in \mathbb{Q}: x^2 < 2$\} does not have a supremum in $\mathbb{Q}$. In this
example, we notice that this subset is bounded and has infinite elements. The rational numbers in this subset can get
\textbf{infinitely closer} to $\pm \sqrt{2}$ but can never reach them. The property of \textbf{``getting infinitely
  closer''} can give us the hint of finding the appropriate binary relation $\prec$.

% ------------------------------
\subsubsection{Correct solution}
% ------------------------------

With the help of the standard ordering relation $<$ in $\mathbb{N}$, let's define the nonstandard ordering relation
$\prec$ in $\mathbb{N}$ as follows. $\forall x, y \in \mathbb{N}$:
\begin{enumerate}
  \item $IsOdd(x) \land IsEven(y) \rightarrow x \prec y$
  \item $IsEven(x) \land IsOdd(y) \rightarrow x \succ y$
  \item $IsOdd(x) \land IsOdd(y) \rightarrow (x < y \rightarrow x \prec y) \lor (x = y \rightarrow x = y) \lor
          (x > y \rightarrow x \succ y)$
  \item $IsEven(x) \land IsEven(y) \rightarrow (x < y \rightarrow x \prec y) \lor (x = y \rightarrow x = y) \lor
          (x > y \rightarrow x \succ y)$
\end{enumerate}

In sum:
\begin{enumerate}
  \item All the odd numbers are less than the even numbers, e.g., $19337 \prec 2$.
  \item Within all the odd (or even) numbers, the numbers are ordered in the same way as the standard ordering, e.g.,
        $1 \prec 3 \prec 5 \prec 7 \ldots$, $2 \prec 4 \prec 6 \prec 8 \ldots$
\end{enumerate}

Then we examine the proper subset $A = \{ x \in \mathbb{N}: IsOdd(x) \}$:
\begin{itemize}
  \item (a): $IsEven(2) \land 2 \notin A$
  \item (b): $\forall e \in \mathbb{N}. (IsEven(e) \rightarrow (\forall a \in A. (a \prec e)))$
  \item (c): $(b) \rightarrow (\forall e \in \mathbb{N}. (IsEven(e) \rightarrow IsUpperBound(e, A, \mathbb{N})))$
  \item (d): $(a) \land (c) \rightarrow IsUpperBound(2, A, \mathbb{N})$
  \item (e): $\forall e \in \mathbb{N}. (IsEven(e) \rightarrow 2 \le e)$
  \item (f): $(c) \land (d) \land (e) \rightarrow sup A = 2$
\end{itemize}

Therefore, $A$ under this $\prec$ relation has $sup A$ but $sup A \notin A$.

% ******************************
\subsection{Exercise 1.1.8}
% ******************************

% ------------------------------
\subsubsection{Description}
% ------------------------------

Let $F := \{ 0, 1, 2 \}$:
\begin{itemize}
  \item a) Prove that there is exactly one way to define addition and multiplication so that $F$ is a field if 0 and 1
        have their usual meaning of (A4) and (M4).
  \item b) Show that $F$ cannot be an ordered field.
\end{itemize}

% ------------------------------
\subsubsection{Proof (a)}
% ------------------------------

The exercise description doesn't say this explicitly but I believe the following are implicated:
\begin{itemize}
  \item $0 = 0$
  \item $1 = 1$
  \item $2 = 2$
  \item $0 \ne 1$
  \item $0 \ne 2$
  \item $1 \ne 2$
\end{itemize}

\textbf{(1a)} Because $F$ is a field, and 0 and 1 have their usual meaning of (A4) and (M4), that means $F$ must satisfy Definition
1.1.5 (A1): $\forall x, y \in F. (x + y \in F)$. Therefore:
\begin{enumerate}
  \item $0 + 0 = 0 \in F$
  \item $0 + 1 = 1 \in F$
  \item $0 + 2 = 2 \in F$
  \item $1 + 0 = 0 + 1 = 1 \in F$ (Definition 1.1.5 (A2))
  \item $1 + 1 = A \in F$
  \item $1 + 2 = B \in F$
  \item $2 + 0 = 2 \in F$
  \item $2 + 1 = 1 + 2 = B \in F$ (Definition 1.1.5 (A2))
  \item $2 + 2 = C \in F$
\end{enumerate}

$A, B, C$ can be one of $0, 1, 2$ independently. Therefore, there are totally $3^3 = 27$ possible assignments:

\begin{table}[H]
  \centering
  \begin{tabular}{|c|c|c|ll}
    \cline{1-3}
    A        & B        & C        &  & \\ [1ex] \cline{1-3}
    0        & 0        & 0        &  & \\ [0.5ex] \cline{1-3}
    0        & 0        & 1        &  & \\ [0.5ex] \cline{1-3}
    0        & 0        & 2        &  & \\ [0.5ex] \cline{1-3}
    0        & 1        & 0        &  & \\ [0.5ex] \cline{1-3}
    0        & 1        & 1        &  & \\ [0.5ex] \cline{1-3}
    0        & 1        & 2        &  & \\ [0.5ex] \cline{1-3}
    $\ldots$ & $\ldots$ & $\ldots$ &  & \\ [0.5ex] \cline{1-3}
  \end{tabular}
  \caption{Possible assignments for A, B, C}
  \label{table:1.1.8-1a}
\end{table}

We just don't know which possible assignment can make $F$ a field and will need to figure that out.

\textbf{(1b)} Let's then examine Definition 1.1.5 (A5) because (A5) is simpler than (A3). (A5) says $\forall x \in F,
  \exists y \in F. (x + y = 0)$.

If we examine the expressions in (1a), we can see that:
\begin{itemize}
  \item $\because 0 + 0 = 0, 0 + 1 = 1 \ne 0, 0 + 2 = 2 \ne 0$, $\therefore$ $0$ is the ``y'' of itself.
  \item $\because 1 + 0 = 1 \ne 0, 1 + 1 = A, 1 + 2 = B$, $\therefore$ one of $A$ or $B$ must be $0$.
  \item $\because 2 + 0 = 2 \ne 0, 2 + 1 = 1 + 2 = B, 2 + 2 = C$, $\therefore$ one of $B$ or $C$ must be $0$.
\end{itemize}

That reduces table \ref{table:1.1.8-1a} to only 11 possibilities:

\begin{table}[H]
  \centering
  \begin{tabular}{|c|c|c|ll}
    \cline{1-3}
    A & B & C &  & \\ [1ex] \cline{1-3}
    0 & 0 & 0 &  & \\ [0.5ex] \cline{1-3}
    0 & 0 & 1 &  & \\ [0.5ex] \cline{1-3}
    0 & 0 & 2 &  & \\ [0.5ex] \cline{1-3}
    0 & 1 & 0 &  & \\ [0.5ex] \cline{1-3}
    0 & 2 & 0 &  & \\ [0.5ex] \cline{1-3}
    1 & 0 & 0 &  & \\ [0.5ex] \cline{1-3}
    1 & 0 & 1 &  & \\ [0.5ex] \cline{1-3}
    1 & 0 & 2 &  & \\ [0.5ex] \cline{1-3}
    2 & 0 & 0 &  & \\ [0.5ex] \cline{1-3}
    2 & 0 & 1 &  & \\ [0.5ex] \cline{1-3}
    2 & 0 & 2 &  & \\ [0.5ex] \cline{1-3}
  \end{tabular}
  \caption{Possible assignments for A, B, C (reduced)}
  \label{table:1.1.8-1b}
\end{table}

\textbf{(1c)} Because $F$ is a field, $F$ must also satisfy Definition 1.1.5 (A3): $\forall x, y, z \in F. ((x + y) + z
  = x + (y + z))$. There are totally $3^3 = 27$ possible cases:
\begin{enumerate}
  \item $(0 + 0) + 0 = 0 + (0 + 0) = 0$
  \item $(0 + 0) + 1 = 0 + (0 + 1) = 1$
  \item $(0 + 0) + 2 = 0 + (0 + 2) = 2$
  \item $(0 + 1) + 0 = 0 + (1 + 0) = 1$
  \item $(0 + 1) + 1 = 0 + (1 + 1) = A$
  \item $(0 + 1) + 2 = 0 + (1 + 2) = B$
  \item $(0 + 2) + 0 = 0 + (2 + 0) = 2$
  \item $(0 + 2) + 1 = 0 + (2 + 1) = B$
  \item $(0 + 2) + 2 = 0 + (2 + 2) = C$
  \item $(1 + 0) + 0 = 1 + (0 + 0) = 1$
  \item $(1 + 0) + 1 = 1 + (0 + 1) = A$
  \item $(1 + 0) + 2 = 1 + (0 + 2) = B$
  \item $(1 + 1) + 0 = 1 + (1 + 0) = A$
  \item $(1 + 1) + 1 = 1 + (1 + 1) = A + 1 = 1 + A$
  \item $(1 + 1) + 2 = 1 + (1 + 2) \rightarrow A + 2 = 1 + B$
  \item $(1 + 2) + 0 = 1 + (2 + 0) = B$
  \item $(1 + 2) + 1 = 1 + (2 + 1) = B + 1 = 1 + B$
  \item $(1 + 2) + 2 = 1 + (2 + 2) = B + 2 = 1 + C$
  \item $(2 + 0) + 0 = 2 + (0 + 0) = 2$
  \item $(2 + 0) + 1 = 2 + (0 + 1) = B$
  \item $(2 + 0) + 2 = 2 + (0 + 2) = C$
  \item $(2 + 1) + 0 = 2 + (1 + 0) = B$
  \item $(2 + 1) + 1 = 2 + (1 + 1) \rightarrow B + 1 = 2 + A$
  \item $(2 + 1) + 2 = 2 + (1 + 2) = B + 2 = 2 + B$
  \item $(2 + 2) + 0 = 2 + (2 + 0) = C$
  \item $(2 + 2) + 1 = 2 + (2 + 1) \rightarrow C + 1 = 2 + B$
  \item $(2 + 2) + 2 = 2 + (2 + 2) = C + 2 = 2 + C$
\end{enumerate}

Therefore, $A, B, C$ must also satisfy the following under the addition operation:

\begin{empheq}[left=\empheqlbrace]{align}
  &A + 2 = 1 + B \\
  &B + 2 = 1 + C
\end{empheq}

Equation (1.1) implies that $A$ and $B$ can't be $0$ at the same time, because otherwise, the left side would be
$0 + 2 = 2$ and the right side be $1 + 0 = 1$, but we know that $2 \ne 1$. Similarly, equation (1.2) implies that $B$
and $C$ can't be $0$ at the same time.

Equation (1.1) also implies that: when $B = 0$, $A \ne 1$. Suppose $B = 0 \land A = 1$, then the left side would be
$1 + 2 = B = 0$ and the right side be $1 + 0 = 1$, but we know $0 \ne 1$.

Equation (1.1) also implies that: when $B = 1$, $A \ne 0$. Suppose $B = 1 \land A = 0$, then the left side would be
$0 + 2 = 2$ and the right side be $1 + 1 = A = 0$, but we know $2 \ne 0$.

These implications can further reduce table \ref{table:1.1.8-1b} down to 3 possibilities:

\begin{table}[H]
  \centering
  \begin{tabular}{|c|c|c|ll}
    \cline{1-3}
    A & B & C &  & \\ [1ex] \cline{1-3}
    0 & 2 & 0 &  & \\ [0.5ex] \cline{1-3}
    2 & 0 & 1 &  & \\ [0.5ex] \cline{1-3}
    2 & 0 & 2 &  & \\ [0.5ex] \cline{1-3}
  \end{tabular}
  \caption{Possible assignments for A, B, C (reduced)}
  \label{table:1.1.8-1c}
\end{table}

Now we can verify each possibility:
\begin{itemize}
  \item $(A = 0, B = 2, C = 0)$ doesn't work: $C = 0 \rightarrow 2 + 2 = 0$. In equation (1.2), the left side would be
        $2 + 2 = C = 0$ and the right side would be $1 + 0 = 1$. But we know $0 \ne 1$.
  \item $(A = 2, B = 0, C = 1)$ works:
        \begin{itemize}
          \item Equation (1.1): Left side is $2 + 2 = C = 1$; right side is $1 + 0 = 1$. No contradiction.
          \item Equation (1.2): Left side is $0 + 2 = 2$; right side is $1 + 1 = A = 2$. No contradiction.
        \end{itemize}
  \item $(A = 2, B = 0, C = 2)$ doesn't work: In equation (1.2), the left side is $0 + 2 = 2$; the right side is
        $1 + 2 = B = 0$. But we know $2 \ne 0$.
\end{itemize}

Therefore, $A = 2, B = 0, C = 1$. In other words: this is the only definition for the addition operation that makes
$F$ a field as defined as follows:
\begin{enumerate}
  \item $0 + 0 = 0 \in F$
  \item $0 + 1 = 1 \in F$
  \item $0 + 2 = 2 \in F$
  \item $1 + 0 = 1 \in F$
  \item $1 + 1 = 2 \in F$
  \item $1 + 2 = 0 \in F$
  \item $2 + 0 = 2 \in F$
  \item $2 + 1 = 1 + 2 = 0 \in F$
  \item $2 + 2 = 1 \in F$
\end{enumerate}

\textbf{(2a)}: Because $F$ is a field, and 0 and 1 have their usual meaning of (A4) and (M4), that means $F$ must
satisfy Definition 1.1.5 (M1): $\forall x, y \in F. (x \cdot y \in F)$. Therefore:
\begin{enumerate}
  \item $0 \cdot 0 = A \in F$
  \item $0 \cdot 1 = 1 \cdot 0 = 0 \in F$ (Definition 1.1.5 (M2), (M4))
  \item $0 \cdot 2 = B \in F$
  \item $1 \cdot 0 = 0 \in F$
  \item $1 \cdot 1 = 1 \in F$
  \item $1 \cdot 2 = 2 \in F$
  \item $2 \cdot 0 = 0 \cdot 2 = B \in F$
  \item $2 \cdot 1 = 1 \cdot 2 = 2 \in F$
  \item $2 \cdot 2 = C \in F$
\end{enumerate}

$A, B, C$ can be one of $0, 1, 2$ independently. Therefore, there are totally $3^3 = 27$ possible assignments:

\begin{table}[H]
  \centering
  \begin{tabular}{|c|c|c|ll}
    \cline{1-3}
    A        & B        & C        &  & \\ [1ex] \cline{1-3}
    0        & 0        & 0        &  & \\ [0.5ex] \cline{1-3}
    0        & 0        & 1        &  & \\ [0.5ex] \cline{1-3}
    0        & 0        & 2        &  & \\ [0.5ex] \cline{1-3}
    0        & 1        & 0        &  & \\ [0.5ex] \cline{1-3}
    0        & 1        & 1        &  & \\ [0.5ex] \cline{1-3}
    0        & 1        & 2        &  & \\ [0.5ex] \cline{1-3}
    $\ldots$ & $\ldots$ & $\ldots$ &  & \\ [0.5ex] \cline{1-3}
  \end{tabular}
  \caption{Possible assignments for A, B, C}
  \label{table:1.1.8-2a}
\end{table}

We just don't know which possible assignment can make $F$ a field and will need to figure that out.

\textbf{(2b)} Let's then examine Definition 1.1.5 (M5) because (M5) is simpler than (M3). (M5) says $\forall x \in F,
  \exists y \in F. (x \cdot y = 1)$.

If we examine the expressions in (2a), we can see that:
\begin{itemize}
  \item $\because 0 \cdot 0 = A; 0 \cdot 1 = 0 \ne 1; 0 \cdot 2 = B$, $\therefore$ one of $A$ and $B$ must be 1.
  \item $\because 1 \cdot 0 = 0 \ne 1; 1 \cdot 1 = 1; 1 \cdot 2 = 2 \ne 1$, $\therefore$ 1 is the ``y'' of itself.
  \item $\because 2 \cdot 0 = B; 2 \cdot 1 = 2 \ne 1; 2 \cdot 2 = C$, $\therefore$ one of $B$ and $C$ must be 1.
\end{itemize}

That reduces table \ref{table:1.1.8-2a} to only 11 possibilities:

\begin{table}[H]
  \centering
  \begin{tabular}{|c|c|c|ll}
    \cline{1-3}
    A & B & C &  & \\ [1ex] \cline{1-3}
    0 & 1 & 0 &  & \\ [0.5ex] \cline{1-3}
    0 & 1 & 1 &  & \\ [0.5ex] \cline{1-3}
    0 & 1 & 2 &  & \\ [0.5ex] \cline{1-3}
    1 & 0 & 1 &  & \\ [0.5ex] \cline{1-3}
    1 & 1 & 0 &  & \\ [0.5ex] \cline{1-3}
    1 & 1 & 1 &  & \\ [0.5ex] \cline{1-3}
    1 & 1 & 2 &  & \\ [0.5ex] \cline{1-3}
    1 & 2 & 1 &  & \\ [0.5ex] \cline{1-3}
    2 & 1 & 0 &  & \\ [0.5ex] \cline{1-3}
    2 & 1 & 1 &  & \\ [0.5ex] \cline{1-3}
    2 & 1 & 2 &  & \\ [0.5ex] \cline{1-3}
  \end{tabular}
  \caption{Possible assignments for A, B, C (reduced)}
  \label{table:1.1.8-2b}
\end{table}

\textbf{(2c)} Because $F$ is a field, $F$ must also satisfy Definition 1.1.5 (M3): $\forall x, y, z \in F. ((x \cdot y)
  \cdot z = x \cdot (y \cdot z))$. There are totally $3^3 = 27$ possible cases:
\begin{enumerate}
  \item $(0 \cdot 0) \cdot 0 = 0 \cdot (0 \cdot 0) = 0 \cdot A$
  \item $(0 \cdot 0) \cdot 1 = 0 \cdot (0 \cdot 1) = A$
  \item $(0 \cdot 0) \cdot 2 = 0 \cdot (0 \cdot 2) \rightarrow A \cdot 2 = 0 \cdot B$
  \item $(0 \cdot 1) \cdot 0 = 0 \cdot (1 \cdot 0) = A$
  \item $(0 \cdot 1) \cdot 1 = 0 \cdot (1 \cdot 1) = 0$
  \item $(0 \cdot 1) \cdot 2 = 0 \cdot (1 \cdot 2) = B$
  \item $(0 \cdot 2) \cdot 0 = 0 \cdot (2 \cdot 0) = 0 \cdot B$
  \item $(0 \cdot 2) \cdot 1 = 0 \cdot (2 \cdot 1) = B$
  \item $(0 \cdot 2) \cdot 2 = 0 \cdot (2 \cdot 2) \rightarrow B \cdot 2 = 0 \cdot C$
  \item $(1 \cdot 0) \cdot 0 = 1 \cdot (0 \cdot 0) = A$
  \item $(1 \cdot 0) \cdot 1 = 1 \cdot (0 \cdot 1) = 0$
  \item $(1 \cdot 0) \cdot 2 = 1 \cdot (0 \cdot 2) = B$
  \item $(1 \cdot 1) \cdot 0 = 1 \cdot (1 \cdot 0) = 0$
  \item $(1 \cdot 1) \cdot 1 = 1 \cdot (1 \cdot 1) = 1$
  \item $(1 \cdot 1) \cdot 2 = 1 \cdot (1 \cdot 2) = 2$
  \item $(1 \cdot 2) \cdot 0 = 1 \cdot (2 \cdot 0) = B$
  \item $(1 \cdot 2) \cdot 1 = 1 \cdot (2 \cdot 1) = 2$
  \item $(1 \cdot 2) \cdot 2 = 1 \cdot (2 \cdot 2) = C$
  \item $(2 \cdot 0) \cdot 0 = 2 \cdot (0 \cdot 0) \rightarrow B \cdot 0 = 2 \cdot A \rightarrow A \cdot 2 = 0 \cdot B$
  \item $(2 \cdot 0) \cdot 1 = 2 \cdot (0 \cdot 1) = B$
  \item $(2 \cdot 0) \cdot 2 = 2 \cdot (0 \cdot 2) = B \cdot 2$
  \item $(2 \cdot 1) \cdot 0 = 2 \cdot (1 \cdot 0) = B$
  \item $(2 \cdot 1) \cdot 1 = 2 \cdot (1 \cdot 1) = 2$
  \item $(2 \cdot 1) \cdot 2 = 2 \cdot (1 \cdot 2) = C$
  \item $(2 \cdot 2) \cdot 0 = 2 \cdot (2 \cdot 0) \rightarrow C \cdot 0 = 2 \cdot B \rightarrow B \cdot 2 = 0 \cdot C$
  \item $(2 \cdot 2) \cdot 1 = 2 \cdot (2 \cdot 1) = C$
  \item $(2 \cdot 2) \cdot 2 = 2 \cdot (2 \cdot 2) = 2 \cdot C$
\end{enumerate}

Therefore, $A, B, C$ must also satisfy the following under the multiplication operation:

\begin{empheq}[left=\empheqlbrace]{align}
  &A \cdot 2 = 0 \cdot B \\
  &B \cdot 2 = 0 \cdot C
\end{empheq}

Equation (1.3) implies that $A$ and $B$ can't be 1 at the same time, because otherwise, the left side would be $1 \cdot
  2 = 2$ while the right side be $0 \cdot 1 = 0$, but we know that $2 \ne 0$. Similarly, equation (1.4) implies that
$B$ and $C$ can't be 1 at the same time.

Equation (1.3) also implies that: when $B = 1$, $A \ne 0$. Suppose $B = 1 \land A = 0$, then the left side would be
$0 \cdot 2 = B = 1$ and the right side be $0 \cdot 1 = 0$. But we know $1 \ne 0$.

Equation (1.3) also implies that: when $B = 0$, $A \ne 1$. Suppose $B = 0 \land A = 1$, then the left side would be
$1 \cdot 2 = 2$ and the right side be $0 \cdot 0 = A = 1$. But we know $2 \ne 1$.

These implications can further reduce table \ref{table:1.1.8-2b} down to 3 possibilities:

\begin{table}[H]
  \centering
  \begin{tabular}{|c|c|c|ll}
    \cline{1-3}
    A & B & C &  & \\ [1ex] \cline{1-3}
    1 & 2 & 1 &  & \\ [0.5ex] \cline{1-3}
    2 & 1 & 0 &  & \\ [0.5ex] \cline{1-3}
    2 & 1 & 2 &  & \\ [0.5ex] \cline{1-3}
  \end{tabular}
  \caption{Possible assignments for A, B, C (reduced)}
  \label{table:1.1.8-2c}
\end{table}

Now we can verify each possibility:
\begin{itemize}
  \item $(A = 1, B = 2, C = 1)$ doesn't work: In equation (1.4), the left side is $2 \cdot 2 = C = 1$ and the right side
        be $0 \cdot 1 = 0$. But we know $1 \ne 0$.
  \item $(A = 2, B = 1, C = 0)$ works:
        \begin{itemize}
          \item Equation (1.3): Left side is $2 \cdot 2 = C = 0$; right side is $0 \cdot 1 = 0$. No contradiction.
          \item Equation (1.4): Left side is $1 \cdot 2 = 2$; right side is $0 \cdot 0 = A = 2$. No contradiction.
        \end{itemize}
  \item $(A = 2, B = 1, C = 2)$ doesn't work: In equation (1.3), the left side is $2 \cdot 2 = C = 2$ and the right side
        be $0 \cdot 1 = 0$. But we know $2 \ne 0$.
\end{itemize}

Therefore, $A = 2, B = 1, C = 0$. In other words, this is the only definition for the multiplication operation that
makes $F$ a field as defined as follows:
\begin{enumerate}
  \item $0 \cdot 0 = 2 \in F$
  \item $0 \cdot 1 = 0 \in F$
  \item $0 \cdot 2 = 1 \in F$
  \item $1 \cdot 0 = 0 \in F$
  \item $1 \cdot 1 = 1 \in F$
  \item $1 \cdot 2 = 2 \in F$
  \item $2 \cdot 0 = 1 \in F$
  \item $2 \cdot 1 = 2 \in F$
  \item $2 \cdot 2 = 0 \in F$
\end{enumerate}

\textbf{(3)} Given the addition and multiplication defined above, we need to verify that Definition 1.1.5 (D) holds,
which says $\forall x y, z \in F. (x \cdot (y + z) = x \cdot y + x \cdot z)$:
\begin{enumerate}
  \item $0 \cdot (0 + 0) = 0 \cdot 0 + 0 \cdot 0$
  \item $0 \cdot (0 + 1) = 0 \cdot 0 + 0 \cdot 1$ (\colorbox{red}{wrong})
  \item $0 \cdot (0 + 2) = 0 \cdot 0 + 0 \cdot 2$ (\colorbox{red}{wrong})
  \item $0 \cdot (1 + 0) = 0 \cdot 1 + 0 \cdot 0$ (\colorbox{yellow}{unverified})
  \item $0 \cdot (1 + 1) = 0 \cdot 1 + 0 \cdot 1$ (\colorbox{yellow}{unverified})
  \item $0 \cdot (1 + 2) = 0 \cdot 1 + 0 \cdot 2$ (\colorbox{yellow}{unverified})
  \item $0 \cdot (2 + 0) = 0 \cdot 2 + 0 \cdot 0$ (\colorbox{yellow}{unverified})
  \item $0 \cdot (2 + 1) = 0 \cdot 2 + 0 \cdot 1$ (\colorbox{yellow}{unverified})
  \item $0 \cdot (2 + 2) = 0 \cdot 2 + 0 \cdot 2$ (\colorbox{yellow}{unverified})
  \item $1 \cdot (0 + 0) = 1 \cdot 0 + 1 \cdot 0$ (\colorbox{yellow}{unverified})
  \item $1 \cdot (0 + 1) = 1 \cdot 0 + 1 \cdot 1$ (\colorbox{yellow}{unverified})
  \item $1 \cdot (0 + 2) = 1 \cdot 0 + 1 \cdot 2$ (\colorbox{yellow}{unverified})
  \item $1 \cdot (1 + 0) = 1 \cdot 1 + 1 \cdot 0$ (\colorbox{yellow}{unverified})
  \item $1 \cdot (1 + 1) = 1 \cdot 1 + 1 \cdot 1$ (\colorbox{yellow}{unverified})
  \item $1 \cdot (1 + 2) = 1 \cdot 1 + 1 \cdot 2$ (\colorbox{yellow}{unverified})
  \item $1 \cdot (2 + 0) = 1 \cdot 2 + 1 \cdot 0$ (\colorbox{yellow}{unverified})
  \item $1 \cdot (2 + 1) = 1 \cdot 2 + 1 \cdot 1$ (\colorbox{yellow}{unverified})
  \item $1 \cdot (2 + 2) = 1 \cdot 2 + 1 \cdot 2$ (\colorbox{yellow}{unverified})
  \item $2 \cdot (0 + 0) = 2 \cdot 0 + 2 \cdot 0$ (\colorbox{yellow}{unverified})
  \item $2 \cdot (0 + 1) = 2 \cdot 0 + 2 \cdot 1$ (\colorbox{yellow}{unverified})
  \item $2 \cdot (0 + 2) = 2 \cdot 0 + 2 \cdot 2$ (\colorbox{yellow}{unverified})
  \item $2 \cdot (1 + 0) = 2 \cdot 1 + 2 \cdot 0$ (\colorbox{yellow}{unverified})
  \item $2 \cdot (1 + 1) = 2 \cdot 1 + 2 \cdot 1$ (\colorbox{yellow}{unverified})
  \item $2 \cdot (1 + 2) = 2 \cdot 1 + 2 \cdot 2$ (\colorbox{yellow}{unverified})
  \item $2 \cdot (2 + 0) = 2 \cdot 2 + 2 \cdot 0$ (\colorbox{yellow}{unverified})
  \item $2 \cdot (2 + 1) = 2 \cdot 2 + 2 \cdot 1$ (\colorbox{yellow}{unverified})
  \item $2 \cdot (2 + 2) = 2 \cdot 2 + 2 \cdot 2$ (\colorbox{yellow}{unverified})
\end{enumerate}

\colorbox{red}{TODO(ywen)}: The law of distribution isn't satisfies under the addition and
multiplication I found above. What happened?

% ------------------------------
\subsubsection{Proof (b)}
% ------------------------------

\colorbox{red}{TODO(ywen)}: Prove (b).

\end{document}
