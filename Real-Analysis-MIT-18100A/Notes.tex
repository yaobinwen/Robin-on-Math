% Regarding `oneside` (https://stackoverflow.com/a/8371473/630364):
%
% `oneside` removes the blank pages between chapters.
% "Note that this method make the margins of all the pages the same. In
% `twoside`, the margins are different for the odd and the even pages".
\documentclass[12pt, letterpaper, oneside]{book}
\usepackage{amsfonts}
\usepackage{amsmath}
\usepackage{amssymb}
\usepackage{csquotes}
\usepackage{float}
\usepackage{hyperref}
\usepackage[letterpaper, textwidth=7.5in, textheight=8in]{geometry}
\hypersetup{
  colorlinks=true,
  linkcolor=blue,
  filecolor=magenta,
  urlcolor=blue,
}
\usepackage{parskip}

\title{
  Notes on \textit{Basic Analysis I: Introduction to Real Analysis, Volume 1}
}
\author{Yaobin Wen}
\date{July 2023}

\begin{document}

\maketitle
\tableofcontents

\chapter*{Overview}
\addcontentsline{toc}{chapter}{Overview}

This document contains my study notes of the textbook \textit{Basic Analysis I:
Introduction to Real Analysis, Volume 1}. I use it for a few purposes:

\begin{enumerate}
  \item As a reference to quickly refresh my memory on the subjects.
  \item Keep the notes to help me understand the text that is not obvious for
    me to comprehend.
\end{enumerate}

% =============================================================================
%
% Chapter: Review of Real Numbers
%
% =============================================================================

\chapter*{Introduction}
\addcontentsline{toc}{chapter}{Introduction}

% =============================================================================
\section{About this book}
% =============================================================================

(Nothing to write down.)

% =============================================================================
\section{About analysis}
% =============================================================================

The \textbf{most important difference} between analysis and algebra:
\begin{itemize}
  \item In algebra, we prove \textbf{equalities} directly.
  \item In analysis, we usually prove \textbf{inequalities} and we prove those
    inequalities by estimating.
\end{itemize}

% =============================================================================
\section{Basic set theory}
% =============================================================================

% ******************************
\subsection{Sets}
% ******************************

\textbf{Definition 0.3.1.} A \textit{set} is a collection of objects called
\textit{elements} or \textit{members}. A set with no objects is called the
\textit{empty set} and is denoted by $\emptyset$ (or sometimes by $\{\}$).

The \textbf{universe} is the set that contains only the elements that we are
interested in. It is generally understood from context and is not explicitly
mentioned.

The various set-related notations:
\begin{itemize}
  \item \textbf{Membership}:
    \begin{itemize}
      \item[$\bullet$] $a \in A$: the element $a$ belongs to the set $A$
      \item[$\bullet$] $a \notin A$: the element $a$ does not belong to the set
        $A$.
    \end{itemize}
  \item \textbf{Subset}:
    \begin{itemize}
      \item Used in this book: $A \subset B$ (at times: $B \supset A$)
      \item Alternatively: Some people prefer to use $A \subseteq B$ to denote
        ``A is a subset of B''.
    \end{itemize}
  \item \textbf{Equality}:
    \begin{itemize}
      \item[$\bullet$] $A = B$: A and B contain exactly the same elements.
      \item[$\bullet$] $A \neq B$
    \end{itemize}
  \item \textbf{Proper subset}:
    \begin{itemize}
      \item Used in this book: $A \subsetneq B$: $A \subset B$ and $A \neq B$.
      \item Alternatively: Some people prefer to use $A \subset B$ to denote
        ``A is a proper subset of B''. Note this can be confused with the
        notation of ``subset'' so consistency of usage is important.
    \end{itemize}
  \item \textbf{Set building notation}: $\{x \in A: P(x)\}$ refers to a subset
    of the set $A$ containing all the elements of $A$ that satisfy the property
    $P(x)$.
  \item \textbf{Natural numbers}: $\mathbb{N} := \{1, 2, 3, \ldots \}$
  \item \textbf{Integers}: $\mathbb{Z} := \{0, -1, 1, -2, 2, \ldots \}$
  \item \textbf{Rational numbers}: $\mathbb{Q} := \{\frac{m}{n}: m,n \in
    \mathbb{Z}, n \neq 0 \}$
  \item \textbf{Real numbers}: $\mathbb{R}$
  \item \textbf{Union}: $A \cup B := \{x: x \in A \ or \ x \in B\}$
  \item \textbf{Intersection}: $A \cap B := \{x: x \in A \ and \ x \in B\}$
  \item \textbf{Complement}:
    \begin{itemize}
      \item $A \setminus B := \{x: x \in A \ and \ x \notin B\}$
      \item $B^c$: If the set $A$ can be understood from the context. Here ``c''
        is the first letter of the word ``complement'', not a set $c$.
    \end{itemize}
  \item \textbf{Disjoint}: $A$ and $B$ are said to be \textit{disjoint} if $A
    \cap B = \emptyset$.
  \item \textbf{Union of an infinite collection of sets}:
    \[
      \bigcup_{n=1}^{\infty} A_n := \{x: x \in A_i, \exists i \in \mathbb{N}\}
    \]
    Note the use of $\exists$: Because this is the union of the sets $A_i$,
    by definition, it means that for any element $x$ in the final union, $x$
    belongs to at least one of the sets $A_i$, but it may or may not belong to
    all of them. Therefore, when we describe the membership of $x$, we must
    express the meaning of ``may or may not''.
  \item \textbf{Intersection of an infinite collection of sets}:
    \[
      \bigcap_{n=1}^{\infty} A_n := \{x: x \in A_i, \forall i \in \mathbb{N}\}
    \]
    The use of $\forall$ means that any element $x$ in the final set belongs
    to all of the sets $A_i$.
  \item \textbf{Unions/Intersection of sets with two indices}:
    \[
      \bigcup_{n=1}^{\infty} \bigcup_{m=1}^{\infty} A_{n,m} =
      \bigcup_{n=1}^{\infty} \Bigl(\bigcup_{m=1}^{\infty} A_{n,m}\Bigr)
    \]
    \[
      \bigcap_{n=1}^{\infty} \bigcap_{m=1}^{\infty} A_{n,m} =
      \bigcap_{n=1}^{\infty} \Bigl(\bigcap_{m=1}^{\infty} A_{n,m}\Bigr)
    \]
  \item \textbf{A more general notation}:
    \[
      \bigcup_{\lambda \in I} A_{\lambda} :=
      \{x: x \in A_{\lambda}, \exists \lambda \in I\}
    \]
    \[
      \bigcap_{\lambda \in I} A_{\lambda} :=
      \{x: x \in A_{\lambda}, \forall \lambda \in I\}
    \]
\end{itemize}

\textbf{Theorem 0.3.5 (DeMorgan)}: Let $A$ be the \textit{universe} and $B, C$
the subsets of $A$. Then:

\begin{itemize}
  \item $(B \cup C)^c$ = $B^c \cap C^c$
  \item $(B \cap C)^c$ = $B^c \cup C^c$
\end{itemize}

or, more generally,

\begin{itemize}
  \item $A \setminus (B \cup C)$ = $(A \setminus B) \cap (A \setminus C)$
  \item $A \setminus (B \cap C)$ = $(A \setminus B) \cup (A \setminus C)$
\end{itemize}

% ******************************
\subsection{Induction}
% ******************************

% ------------------------------
\subsubsection*{Theorem 0.3.6: Principle of induction}
% ------------------------------

Let $P(n)$ be a statement depending on a natural number $n$. Suppose that:

\begin{enumerate}
  \item (basis statement) $P(1)$ is true.
  \item (induction step) If $P(n)$ is true, then $P(n+1)$ is true.
\end{enumerate}

Then $P(n)$ is true for all $n \in \mathbb{N}$.

The textbook [1] offers the following proof:

\begin{displayquote}
  Suppose $S$ is the set of natural numbers $m$ for which $P(m)$ is not true.
  Suppose $S$ is nonempty. Then $S$ has a least element by the well ordering
  property. Let us call $m$ the least element of S. We know $1 \notin S$ by
  assumption. So $m > 1$ and $m - 1$ is a natural number as well. Since $m$ is
  the least element of $S$, we know that $P(m-1)$ is true. But by the induction
  step we see that $P(m-1+1) = P(m)$ is true, contradicting the statement that
  $m \in S$. Therefore, $S$ is empty and $P(n)$ is true for all $n \in
  \mathbb{N}$.
\end{displayquote}

My notes on the proof:

\begin{itemize}
  \item The proof uses contradiction. Therefore, we need to suppose the opposite
    conclusion of the theorem is true. Because the conclusion of the theorem is
    ``$P(n)$ is true for \textbf{all} $n \in \mathbb{N}$'', the opposite
    conclusion is ``there exist some $m \in \mathbb{N}$ that $P(m)$ is false''.
    This is why the first sentence says ``... $m$ for which $P(m)$ is not true''
    and ``$S$ is nonempty'': $S$ must not be empty, because otherwise the
    original conclusion is true and it makes no sense to use proof by
    contradiction.
  \item Regarding $1 \notin S$: The basis statement is $P(1)$ is true; $S$ is
    the set of natural numbers $m$ where $P(m)$ is false. Therefore, if $1 \in
    S$, that means $P(1)$ is false, which contradicts with the conditions of
    the theorem to-be-proven. Note that ``proof by contradiction'' starts with
    the opposite \textbf{conclusion} of the original theorem but it still
    accepts all the conditions of the original theorem. In this case, ``proof
    by contradiction'' still accepts the condition that $P(1)$ is true.
\end{itemize}

% ------------------------------
\subsubsection*{Theorem 0.3.9: Principle of strong induction}
% ------------------------------

Let $P(n)$ be a statement depending on a natural number $n$. Suppose that:

\begin{enumerate}
  \item (basis statement) $P(1)$ is true.
  \item (induction step) If $P(k)$ is true for all $k = 1, 2, \ldots, n$, then
    $P(n+1)$ is true.
\end{enumerate}

Then $P(n)$ is true for all $n \in \mathbb{N}$.

The textbook leaves the proof of this theorem as an exercise, so here is mine:

My proof: We still use the method of ``proof by contradiction''. Suppose $S$ is
the set of natural numbers $i$ for which $P(i)$ is not true and $S$ is not
empty. By the well ordering property, we know $S$ has a least element and let's
call it $m$. Because $P(k)$ is true for all $k = 1, 2, \ldots, n$, we know that
$1, 2, \ldots, n \notin S$. Therefore, $m > n$ and $m-1$ is a natural number,
too. Since $m$ is the least element of $S$, we know $m-1 \notin S$. Because
$P(i)$ is not true for all $i \in S$ and because $m-1 \notin S$, we know
$P(m-1)$ is true. However, by the induction step, we see that $P(m-1+1) = P(m)$
must be true, so $m \notin S$, contradicting the statement that $m \in S$.
Therefore, $S$ must be empty and $P(n)$ is true for all $n \in \mathbb{N}$.

% ******************************
\subsection{Functions}
% ******************************

In order to define functions rigorously, we must define the Cartesian product
firstly.

% ------------------------------
\subsubsection*{Definition 0.3.10 Cartesian product}
% ------------------------------

Let $A$ and $B$ be sets. The \textit{Cartesian product} is the set of tuples
defined as \[ A \times B := \{(x, y): x \in A, y \in B \} \].

\textbf{Note}: ``tuple'' is the same as the ``list'' as we see in ``Linear
Algebra Done Right''.

Examples:
\begin{itemize}
  \item $[0,1] \times [0,1]$ = $\{(0,0), (0,1), (1,0), (1,1)\}$
  \item $[0,1] \times [0,1]$ is also denoted as $[0,1]^2$.
  \item $\mathbb{R}^2$ = $\mathbb{R} \times \mathbb{R}$
\end{itemize}

% ------------------------------
\subsubsection*{Definition 0.3.11 Function}
% ------------------------------

A \textit{function} $f: A \rightarrow B$ is a subset $f$ of $A \times B$ such
that for each $x \in A$, there is a \textbf{unique} $(x,y) \in f$. We then
write $f(x) = y$.

The various terms that are related with functions:
\begin{itemize}
  \item Sometimes $f$ is called a \textbf{mapping} or a \textbf{map}: $f$
    \textbf{maps} $A$ to $B$.
  \item Sometimes the set $f$ is called the \textbf{graph} of the function
    rather than the function itself.
  \item \textbf{domain}: The set $A$ is called the domain of $f$. Sometimes it
    is (confusingly) denoted as $D(f)$.
  \item \textbf{range}: The set $R(f) := \{y \in B: \exists x: f(x) = y\}$ is
    called the range of $f$.
\end{itemize}

% ------------------------------
\subsubsection*{Definition 0.3.13 Direct image and inverse image}
% ------------------------------

Consider a function $f: A \rightarrow B$ and $C \subset A$. Define the
\textit{image} (or \textit{direct image}) of $C$ as \[ f(C) := \{f(x) \in B:
x \in C\} \].

Let $D \subset B$. Define the \textit{inverse image} of $D$ as \[ f^{-1}(D) :=
\{x \in A: f(x) \in D\}\].

\textbf{Note} the definition of inverse image: the elements in $D$ do not have
to have an element in $A$ that maps to it. See the example $f^{-1}(\{a,b,c\})$
in \textbf{Figure 3} in the textbook.

% =============================================================================
%
% Chapter 1: Real Numbers
%
% =============================================================================

\chapter{Real Numbers}

TODO

\chapter*{References}
\addcontentsline{toc}{chapter}{References}

\begin{itemize}
  \item $[1]$ \href{https://ocw.mit.edu/courses/18-100a-real-analysis-fall-2020/resources/mit18_100af20_basic_analysis/}{Lebl, Jiří: \it{Basic Analysis I: Introduction to Real Analysis, Volume 1}}
\end{itemize}

\end{document}
