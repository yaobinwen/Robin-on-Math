% Regarding `oneside` (https://stackoverflow.com/a/8371473/630364):
%
% `oneside` removes the blank pages between chapters.
% "Note that this method make the margins of all the pages the same. In
% `twoside`, the margins are different for the odd and the even pages".
\documentclass[12pt, letterpaper, oneside]{book}
\usepackage{amsfonts}
\usepackage{amsmath}
\usepackage{amssymb}
\usepackage{csquotes}
\usepackage{float}
\usepackage{hyperref}
\usepackage[letterpaper, textwidth=7.5in, textheight=8in]{geometry}
\hypersetup{
  colorlinks=true,
  linkcolor=blue,
  filecolor=magenta,
  urlcolor=blue,
}
\usepackage{parskip}

\title{
  Notes on \textit{Basic Analysis I: Introduction to Real Analysis, Volume 1}
}
\author{Yaobin Wen}
\date{July 2023}

\begin{document}

\maketitle
\tableofcontents

\chapter*{Overview}
\addcontentsline{toc}{chapter}{Overview}

This document contains my study notes of the textbook \textit{Basic Analysis I:
Introduction to Real Analysis, Volume 1}. I use it for a few purposes:

\begin{enumerate}
  \item As a reference to quickly refresh my memory on the subjects.
  \item Keep the notes to help me understand the text that is not obvious for
    me to comprehend.
\end{enumerate}

% =============================================================================
%
% Chapter: Review of Real Numbers
%
% =============================================================================

\chapter*{Introduction}
\addcontentsline{toc}{chapter}{Introduction}

% =============================================================================
\section{About this book}
% =============================================================================

(Nothing to write down.)

% =============================================================================
\section{About analysis}
% =============================================================================

The \textbf{most important difference} between analysis and algebra:
\begin{itemize}
  \item In algebra, we prove \textbf{equalities} directly.
  \item In analysis, we usually prove \textbf{inequalities} and we prove those
    inequalities by estimating.
\end{itemize}

% =============================================================================
\section{Basic set theory}
% =============================================================================

% ******************************
\subsection{Sets}
% ******************************

\textbf{Definition 0.3.1.} A \textit{set} is a collection of objects called
\textit{elements} or \textit{members}. A set with no objects is called the
\textit{empty set} and is denoted by $\emptyset$ (or sometimes by $\{\}$).

The \textbf{universe} is the set that contains only the elements that we are
interested in. It is generally understood from context and is not explicitly
mentioned.

The various set-related notations:
\begin{itemize}
  \item \textbf{Membership}:
    \begin{itemize}
      \item[$\bullet$] $a \in A$: the element $a$ belongs to the set $A$
      \item[$\bullet$] $a \notin A$: the element $a$ does not belong to the set
        $A$.
    \end{itemize}
  \item \textbf{Subset}:
    \begin{itemize}
      \item Used in this book: $A \subset B$ (at times: $B \supset A$)
      \item Alternatively: Some people prefer to use $A \subseteq B$ to denote
        ``A is a subset of B''.
    \end{itemize}
  \item \textbf{Equality}:
    \begin{itemize}
      \item[$\bullet$] $A = B$: A and B contain exactly the same elements.
      \item[$\bullet$] $A \neq B$
    \end{itemize}
  \item \textbf{Proper subset}:
    \begin{itemize}
      \item Used in this book: $A \subsetneq B$: $A \subset B$ and $A \neq B$.
      \item Alternatively: Some people prefer to use $A \subset B$ to denote
        ``A is a proper subset of B''. Note this can be confused with the
        notation of ``subset'' so consistency of usage is important.
    \end{itemize}
  \item \textbf{Set building notation}: $\{x \in A: P(x)\}$ refers to a subset
    of the set $A$ containing all the elements of $A$ that satisfy the property
    $P(x)$.
  \item \textbf{Natural numbers}: $\mathbb{N} := \{1, 2, 3, \ldots \}$
  \item \textbf{Integers}: $\mathbb{Z} := \{0, -1, 1, -2, 2, \ldots \}$
  \item \textbf{Rational numbers}: $\mathbb{Q} := \{\frac{m}{n}: m,n \in
    \mathbb{Z}, n \neq 0 \}$
  \item \textbf{Real numbers}: $\mathbb{R}$
  \item \textbf{Union}: $A \cup B := \{x: x \in A \ or \ x \in B\}$
  \item \textbf{Intersection}: $A \cap B := \{x: x \in A \ and \ x \in B\}$
  \item \textbf{Complement}:
    \begin{itemize}
      \item $A \setminus B := \{x: x \in A \ and \ x \notin B\}$
      \item $B^c$: If the set $A$ can be understood from the context. Here ``c''
        is the first letter of the word ``complement'', not a set $c$.
    \end{itemize}
  \item \textbf{Disjoint}: $A$ and $B$ are said to be \textit{disjoint} if $A
    \cap B = \emptyset$.
\end{itemize}

% =============================================================================
%
% Chapter 1: Real Numbers
%
% =============================================================================

\chapter{Real Numbers}

TODO

\chapter*{References}
\addcontentsline{toc}{chapter}{References}

\begin{itemize}
  \item $[1]$ \href{https://ocw.mit.edu/courses/18-100a-real-analysis-fall-2020/resources/mit18_100af20_basic_analysis/}{Lebl, Jiří: \it{Basic Analysis I: Introduction to Real Analysis, Volume 1}}
\end{itemize}

\end{document}
