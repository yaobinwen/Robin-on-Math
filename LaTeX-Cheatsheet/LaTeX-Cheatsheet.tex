\documentclass[12pt, letterpaper, oneside]{article}
\usepackage{amsmath}
\usepackage{csquotes}
\usepackage{float}
\usepackage{hyperref}
\usepackage{listings}
\lstset{basicstyle=\ttfamily\footnotesize,breaklines=true}
\usepackage{tikz}
\usepackage{parskip}
\usepackage{xcolor}
% \usetikzlibrary{arrows.meta,decorations.pathmorphing,backgrounds,positioning,fit,petri}

\hypersetup{
  colorlinks=true,
  linkcolor=blue,
  filecolor=magenta,
  urlcolor=blue,
}

\begin{document}

% =============================================================================
\section{Parentheses}
% =============================================================================

\[\Biggl(\biggl(\Bigl(\bigl((hello)\bigr)\Bigr)\biggr)\Biggr)\]

\[\Biggl[\biggl[\Bigl[\bigl[[hello]\bigr]\Bigr]\biggr]\Biggr]\]

\[\Biggl\{\biggl\{\Bigl\{\bigl\{\{hello\}\bigr\}\Bigr\}\biggr\}\Biggr\}\]

\[
  \Biggl \langle \biggl \langle \Bigl \langle \bigl \langle
  \langle
  hello
  \rangle
  \bigr \rangle \Bigr \rangle \biggr \rangle \Biggr \rangle
\]

% =============================================================================
\section{Quotation}
% =============================================================================

This section demonstrates how to display a quote:

\begin{displayquote}
  This is the quoted part.
\end{displayquote}

This paragraph is below the quoted part.

% =============================================================================
\section{Alignment}
% =============================================================================

\begin{align*}
  0     & \in U_1                      \\
  0     & \in U_2                      \\
        & \ldots                       \\
  + \ 0 & \in U_m                      \\
  = \ 0 & \in U_1 + U_2 + \ldots + U_m
\end{align*}

\begin{align*}
  w_1 + w_2 & = (u_{11} + \ldots + u_{m1}) + (u_{12} + \ldots + u_{m2}) \\
            & = u_{11} + \ldots + u_{m1} + u_{12} + \ldots + u_{m2}     \\
            & = (u_{11} + u_{12}) + \ldots + (u_{m1} + u_{m2})
\end{align*}

% =============================================================================
\section{Negation}
% =============================================================================

In some situations, the tag ``not'' can be used to negate the symbol that follows. This is useful when the ``n''
counterpart is not imported by default:

\begin{itemize}
  \item $a \in S$
  \item $a \notin S$
  \item $a \not\in S$
  \item $a \equiv b$
  \item $a \not\equiv b$
\end{itemize}

% =============================================================================
\section{Table}
% =============================================================================

Useful online tool: \href{https://www.tablesgenerator.com/}{Tables Generator}

Table compares lists and sets:
\begin{table}[H]
  \centering
  \begin{tabular}{||c c c ||}
    \hline
               & Lists   & Sets               \\ [0.5ex]
    \hline
    \hline
    Length     & Finite  & Finite or infinite \\
    Order      & Matters & Doesn't matter     \\
    Repetition & Allows  & Doesn't allow      \\ [1ex]
    \hline
  \end{tabular}
  \caption{Compare lists and sets}
  \label{table:lists_sets_comp}
\end{table}

This is a reference to table \ref{table:lists_sets_comp}

\begin{table}[H]
  \centering
  \begin{tabular}{|c|c|c|ll}
    \cline{1-3}
    p & q & $p \rightarrow q$ &  & \\ [1ex] \cline{1-3}
    F & F & T                 &  & \\ [0.5ex] \cline{1-3}
    F & T & T                 &  & \\ [0.5ex] \cline{1-3}
    T & F & F                 &  & \\ [0.5ex] \cline{1-3}
    T & T & T                 &  & \\ [0.5ex] \cline{1-3}
  \end{tabular}
  \caption{Truth table for $p \rightarrow q$}
\end{table}

% =============================================================================
\section{Code listings}
% =============================================================================

A Python code listing:

\begin{lstlisting}[language=Python]
def say_hello():
    print("Hello, Python!")

say_hello()
\end{lstlisting}

A C code listing:

\begin{lstlisting}[language=C]
  #include <stdio.h>
  int main(int argc, char * argv[]) {
    printf("Hello, world!\n");
    return 0;
  }
\end{lstlisting}

% =============================================================================
\section{Colors}
% =============================================================================

\colorbox{red!100}{100\% red}
\colorbox{red!80}{80\% red}
\colorbox{red!60}{60\% red}
\colorbox{red!40}{40\% red}
\colorbox{red!20}{20\% red}
\colorbox{red!0}{0\% red}

\colorbox{green!100}{100\% green}
\colorbox{blue!100}{\textcolor{yellow}{100\% blue}}
\colorbox{cyan!100}{100\% cyan}
\colorbox{magenta!100}{100\% magenta}
\colorbox{yellow!100}{100\% yellow}
\colorbox{black!100}{\textcolor{yellow}{100\% black}}
\colorbox{gray!100}{100\% gray}
\colorbox{white!100}{100\% white}
\colorbox{darkgray!100}{\textcolor{yellow}{100\% darkgray}}
\colorbox{lightgray!100}{100\% lightgray}
\colorbox{brown!100}{100\% brown}
\colorbox{lime!100}{100\% lime}
\colorbox{olive!100}{100\% olive}
\colorbox{orange!100}{100\% orange}
\colorbox{pink!100}{100\% pink}
\colorbox{purple!100}{\textcolor{yellow}{100\% purple}}
\colorbox{teal!100}{100\% teal}
\colorbox{violet!100}{\textcolor{yellow}{100\% violet}}

\end{document}
